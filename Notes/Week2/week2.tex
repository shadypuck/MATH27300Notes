\documentclass[../notes.tex]{subfiles}

\pagestyle{main}
\renewcommand{\chaptermark}[1]{\markboth{\chaptername\ \thechapter\ (#1)}{}}
\stepcounter{chapter}

\begin{document}




\chapter{Solving Simple ODEs}
\section{Separable ODEs}
\begin{itemize}
    \item \marginnote{10/3:}Do not sit on the left side of the classroom: The sun sucks!
    \item \textbf{Separable} (ODE): An ODE of the form
    \begin{equation*}
        \dv{y}{t} = f(t)g(y)
    \end{equation*}
    where $y$ is a real\footnote{We'll deal with complex functions later.}, unknown, scalar function of $t$.
    \item Solving separable ODEs: Formally, evaluate
    \begin{equation*}
        \int\frac{\dd{y}}{g(y)} = \int f(t)\dd{t}
    \end{equation*}
    \item Rearrange the initial separable ODE to $\dv*{y}{t}\cdot 1/g=f$ and invoke the law of composite differentiation to get
    \begin{equation*}
        \dv{t}\left[ \int_{y_0}^{y(t)}\frac{\dd{w}}{g(w)}-\int_{t_0}^tf(\tau)\dd{\tau} \right] = 0
    \end{equation*}
    \item It follows that
    \begin{equation*}
        \int_{y_0}^{y(t)}\frac{\dd{w}}{g(w)} = \int_{t_0}^tf(\tau)\dd{\tau}
    \end{equation*}
    \item Examples:
    \begin{enumerate}
        \item Exponential growth.
        \begin{itemize}
            \item We have that
            \begin{equation*}
                \dv{y}{t} = ky
            \end{equation*}
            for $k>0$ and $y(0)=y_0>0$.
            \item The solution is
            \begin{align*}
                \frac{1}{y}\cdot\dv{y}{t} &= k\\
                \log y(t)-\log y_0 &= kt\\
                y(t) &= y_0\e[kt]
            \end{align*}
        \end{itemize}
        \item Logistic growth.
        \begin{itemize}
            \item We have that
            \begin{equation*}
                \dv{y}{t} = ky\left( 1-\frac{y}{M} \right)
            \end{equation*}
            for $k,M>0$ and $y(0)=y_0>0$.
            \item The solution is
            \begin{align*}
                \frac{M\dd{y}}{y(M-y)} &= k\dd{t}\\
                \log\frac{y}{M-y}-\log\frac{y_0}{M-y_0} &= kt\\
                \frac{y(M-y_0)}{y_0(M-y)} &= \e[kt]\\
                y\cdot\frac{M-y_0}{y_0} &= (M-y)\e[kt]\\
                y\cdot\frac{M-y_0}{y_0}+y\e[kt] &= M\e[kt]\\
                y\left( \frac{M-y_0}{y_0}+\e[kt] \right) &= M\e[kt]\\
                y\left( \frac{M-y_0+y_0\e[kt]}{y_0} \right) &= M\e[kt]\\
                y\left( \frac{M+y_0(\e[kt]-1)}{y_0} \right) &= M\e[kt]\\
                y(t) &= \frac{My_0\e[kt]}{M+y_0(\e[kt]-1)}
            \end{align*}
            \item Sketches the graph of logistic growth and discusses the turning point (for which there is a formula; zero of the second derivative) as well as general trends.
            \item If $y_0<0$, the solution is not physically meaningful, but it is mathematically insightful.
            \begin{itemize}
                \item When we integrate, the arguments of our logarithms now have absolute values.
                \begin{equation*}
                    \log\left| \frac{y}{M-y} \right|-\log\left| \frac{y_0}{M-y_0} \right| = kt
                \end{equation*}
                \item We need to make sure that the denominator of the final logistic form is never equal to zero, but now that $y_0$ is negative, as $t$ increases, the denominator will approach zero exponentially. It reaches zero when
                \begin{align*}
                    M+y_0(\e[kt]-1) &= 0\\
                    \e[kt] &= -\frac{M}{y_0}+1
                \end{align*}
                In other words, $t_\text{max}=(1/k)\log(1-M/y_0)$ because when $t=t_\text{max}$, the equation blows up.
                \item This is an example of \textbf{finite lifespan}.
            \end{itemize}
            \item If $y_0>M$, then you will exponentially decrease to $M$.
        \end{itemize}
        \item Lotka-Volterra predator-prey model.
        \begin{itemize}
            \item We have that
            \begin{align*}
                r' &= k_1r-awr&
                w' &= -k_2w+bwr
            \end{align*}
            where $r$ is rabbits and $w$ is wolves.
            \item We can rename the variables to
            \begin{equation*}
                \begin{cases}
                    x' = Ax-Bxy\\
                    y' = -Cy+Dxy
                \end{cases}
            \end{equation*}
            \item Dividing, we get
            \begin{align*}
                \frac{x'}{y'} &= \frac{Ax-Bxy}{-Cy+Dxy}\\
                \frac{By-A}{y}y'+\frac{Dx-C}{x}x' &= 0
            \end{align*}
            \item Use the fact that $x,y$ are independent variables, so both terms in the above equation are equal to zero??
            \item Invoke the law of composite differentiation twice and, from the above, know that $0+0=0$, so we can add the two solutions:
            \begin{align*}
                \dv{t}(By(t)-A\log y(t))+\dv{t}(Dx(t)-C\log x(t)) &= 0\\
                By(t)-A\log y(t)+Dx(t)-C\log x(t) &= E
            \end{align*}
            \item Sketches some of the trajectories (they're all closed curves in the $xy$-plane).
            \begin{figure}[h!]
                \centering
                \includegraphics[width=0.4\linewidth]{../ExtFiles/LotkaVolteraSolns.png}
                \caption{Lotka-Volterra solution curves.}
                \label{fig:LotkaVolteraSolns}
            \end{figure}
            \item Properties of the curves:
            \begin{itemize}
                \item The implicit relation which determines them: By the implicit function theorem, the $y$ derivative of the LHS is $B-A/y$ and the $x$-derivative of the LHS is $D-C/x$. When the partial derivatives are equal to zero, $(C/D,A/B)$ becomes interesting. Turning points happen when the $y$-coordinate is $A/B$ or the $x$-coordinate is $C/D$.
            \end{itemize}
        \end{itemize}
    \end{enumerate}
    \item \textbf{Finite lifespan}: Even if the RHS of $\dv*{y}{t}=f(t,y)$ is very regular, the solution can still blow up at some finite time.
    \item Consider the final ODE from the Brachistochrone problem.
    \begin{equation*}
        \dv{y}{x} = \sqrt{\frac{B-y}{y}}
    \end{equation*}
    \begin{itemize}
        \item Finding the \textbf{primitives}.
        \begin{itemize}
            \item What are these "primitives" Shao keeps talking about??
        \end{itemize}
        \item We should have
        \begin{equation*}
            \int\sqrt{\frac{y}{B-y}}\dd{y} = x
        \end{equation*}
        \item Change of variables: $y=B\sin^2\phi$ and $\dd{y}=2B\cos\phi\sin\phi\dd{\phi}$. Thus,
        \begin{equation*}
            \int\sqrt{\frac{y}{B-y}}\dd{y} = \int\frac{\sin\phi}{\cos\phi}\cdot 2B\cos\phi\sin\phi\dd{\phi}
            = 2B\int\sin^2\phi\dd{\phi}
        \end{equation*}
        \item The solution is
        \begin{equation*}
            \begin{cases}
                x = B\phi-\frac{B}{2}\sin(2\phi)+C\\
                y = B\sin^2\phi
            \end{cases}
        \end{equation*}
        \begin{itemize}
            \item This is a parameterization of a cycloid.
        \end{itemize}
    \end{itemize}
    \item Later in the week, we will do the SHM, the pendulum, the Kepler 2-body problem, and the Michaelis-Menten equation.
    \item Separable ODEs are a subset of ODEs of \textbf{exact form}.
    \item ODEs of exact form are of the form
    \begin{equation*}
        g(x,y)\dv{y}{x}+f(x,y) = 0
    \end{equation*}
    where for some $F(x,y)$, $g=\pdv*{F}{y}$, $f=\pdv*{F}{x}$, and partials commute. Equivalently,
    \begin{equation*}
        \pdv{g}{x} = \pdv{f}{y}
    \end{equation*}
    is our necessary and sufficient condition.
    \item By the law of composite differentiation,
    \begin{align*}
        \dv{x}\left[ F(x,y(x)) \right] &= \pdv{F}{x}(x,y(x))+\pdv{F}{y}(x,y(x))\cdot y'(x)\\
        &= f(x,y(x))+g(x,y(x))y'(x)\\
        &= 0
    \end{align*}
    \begin{itemize}
        \item We solve these with an integrating factor $\mu\neq 0$ such that $(\mu g,\mu f)$ satisfy the constraint.
    \end{itemize}
\end{itemize}



\section{Office Hours (Shao)}
\begin{itemize}
    \item \textbf{Primitive}: An antiderivative.
    \item \textbf{Law of composite differentiation}: The chain rule.
    \item Went over how Shao has been applying the law of composite differentiation with respect to separable ODEs:
    \begin{itemize}
        \item Rearrange the initial separable ODE as follows.
        \begin{equation*}
            \frac{1}{g(y)}\cdot\dv{y}{t} = f(t)
        \end{equation*}
        \item Define $\dv*{H}{y}=1/g(y)$. Then, continuing from the above, we have by the law of composite differentiation that
        \begin{align*}
            \dv{H}{y}\cdot\dv{y}{t} &= f(t)\\
            \dv{H}{t} &= f(t)
        \end{align*}
        \item From the definition of $H$, we know that $H(y)=\int_{y_0}^y\dd{w}/g(w)$. We also have from the FTC that $f(t)=\dv{t}\int_{t_0}^tf(\tau)\dd{\tau}$. Thus, continuing from the above, we have that
        \begin{align*}
            \dv{t}(H) &= f(t)\\
            \dv{t}\left[ \int_{y_0}^y\frac{\dd{w}}{g(w)} \right] &= \dv{t}\int_{t_0}^tf(\tau)\dd{\tau}\\
            \dv{t}\left[ \int_{y_0}^{y(t)}\frac{\dd{w}}{g(w)}-\int_{t_0}^tf(\tau)\dd{\tau} \right] &= 0
        \end{align*}
        as desired.
        \item It follows since $y(t_0)=y_0$ that $C=H(y_0)=0$, so we can take the above to
        \begin{equation*}
            \int_{y_0}^{y(t)}\frac{\dd{w}}{g(w)} = \int_{t_0}^tf(\tau)\dd{\tau}
        \end{equation*}
        knowing that our constant of integration is zero.
    \end{itemize}
    \item Take away from Brachistochrone problem: Just an example of a BDE; we won't have to answer questions on it.
\end{itemize}



\section{ODEs of Exact Form}
\begin{itemize}
    \item \marginnote{10/5:}Last time, we discussed separable ODEs.
    \item Today, we will study \textbf{exact form} equations, as discussed last class.
    \item \textbf{Exact form} (ODE): An ODE of the form
    \begin{equation*}
        g(x,y)\dv{y}{x}+f(x,y) = 0
    \end{equation*}
    where
    \begin{equation*}
        \pdv{g}{x} = \pdv{f}{y}
    \end{equation*}
    \item For equations of this form, there exists $F(x,y)$ such that
    \begin{align*}
        \pdv{F}{x} &= f&
        \pdv{F}{y} &= g&
        F(x,y(x)) &= C
    \end{align*}
    for some $C\in\R$.
    \item To solve equations of this form, we need an \textbf{integrating factor}.
    \item \textbf{Integrating factor}: A number or function $\mu$ such that
    \begin{align*}
        \mu g\dv{y}{x}+\mu f &= 0&
        \pdv{x}(\mu g) &= \pdv{y}(\mu f)
    \end{align*}
    \item The solution to linear homogeneous equations of the form $\dv*{y}{t}=p(t)y$ is
    \begin{equation*}
        y(t) = y_0\exp[\int_{t_0}^tp(\tau)\dd\tau]
    \end{equation*}
    \item Recall that $\e[a+ib]=\e[a](\cos b+i\sin b)$, so
    \begin{align*}
        \e[ix] &= \cos x+i\sin x&
        \cos x &= \frac{1}{2}(\e[ix]+\e[-ix])&
        \sin x &= \frac{1}{2i}(\e[ix]-\e[-ix])
    \end{align*}
    \item Example: If $y'=ky$, then $y'=-\lambda y$.
    \item If we have an inhomogeneous linear equation $\dv*{y}{t}=p(t)y+f(t)$, then
    \begin{equation*}
        \dv{y}{t}-py-f = 0
    \end{equation*}
    but
    \begin{equation*}
        0 = \dv{t}(1) \neq \dv{y}(-p(t)y-f(t))
    \end{equation*}
    \item We wish to find an integrating factor $\mu(t,y)$ such that
    \begin{equation*}
        \mu(t,y)\dv{y}{t}-\mu(t,y)p(t)y-\mu(t,y)f(t) = 0
    \end{equation*}
    and
    \begin{equation*}
        \dv{t}(\mu) = \dv{y}(-\mu py-\mu f)
    \end{equation*}
    \item Solution: Take $\mu$ to be a function of $t$, alone. Then
    \begin{equation*}
        \mu'(t) = \dv{y}(-\mu py-\mu f) = -\mu(t)p(t)
    \end{equation*}
    and we now have a homogeneous linear equation with solution
    \begin{equation*}
        \mu(t) = \exp[-\int_{t_0}^tp(\tau)\dd\tau]
    \end{equation*}
    \begin{itemize}
        \item If we let $P(t)=\int_{t_0}^tp(\tau)\dd\tau$, then
        \begin{align*}
            \e[-P(t)]y'(t)-p(t)y(t)\e[-P(t)] &= \e[-P(t)]f(t)\\
            \dv{t}(\e[-P(t)]y(t)) &= \e[-P(t)]f(t)\\
            \e[-P(t)]y(t) &= \int_{t_0}^t\e[-P(\tau)]f(\tau)\dd\tau
        \end{align*}
        \item Thus, we finally have the solution to the inhomogeneous problem as follows: The IVP $y'=py+f$, $y(t_0)=y_0$ has solution
        \begin{equation*}
            y(t) = y_0\e[P(t)-P(t_0)]+\e[P(t)]\int_0^t\e[-P(\tau)]f(\tau)\dd\tau
        \end{equation*}
        where $P$ is any anti-derivative of $p$.
    \end{itemize}
    \item In particular, when $p(t)=a$, we get the \textbf{Duhamel formula} (which we should memorize).
    \item \textbf{Duhamel formula}: The following equation, which is the solution to an inhomogeneous linear equation with $p(t)=a$.
    \begin{equation*}
        y(t) = y_0\e[a(t-t_0)]+\int_{t_0}^t\e[a(t-\tau)]f(\tau)\dd\tau
    \end{equation*}
    \begin{itemize}
        \item Important for computing forced oscillation.
    \end{itemize}
    \item Inspecting the inhomogeneous solution.
    \begin{itemize}
        \item The first term is the solution to the homogeneous form. The second term deals with the initial value.
    \end{itemize}
    \item Given an inhomogeneous equation, you can always write its solution as the combination of the solution to the homogeneous problem plus a particular solution, i.e.,
    \begin{equation*}
        y = y_h+y_p
    \end{equation*}
    \begin{itemize}
        \item "The general solution equals the homogeneous solution plus a particular solution."
        \item This is related to linear algebra, where the solution to $Ax=b$ is a particular solution $x_p$ plus any vector $x\in\ker A$.
        \item Thus, this idea will reappear in the theory of systems of linear ODEs.
    \end{itemize}
    \item We now look at systems of linear ODEs.
    \item Consider the harmonic oscillator: A particle of mass $m$ connected to an ideal spring (obeys Hooke's law) with no friction or gravity.
    \begin{itemize}
        \item Newton's second law: The acceleration is proportional to the restoring force.
        \item Hooke's law: The restoring force is of magnitude $kx$ in the opposite direction to the displacement.
        \item Thus, the ODE is of the form
        \begin{equation*}
            x'' = -\frac{k}{m}x
        \end{equation*}
        \item However, if there is damping (which will be proportional to the velocity), then the ODE is of the form
        \begin{equation*}
            x''+\frac{b}{m}x'+\frac{k}{m}x = 0
        \end{equation*}
    \end{itemize}
    \item Consider an ODE of the form
    \begin{equation*}
        y''+ay'+by = 0
    \end{equation*}
    for $a,b\in\C$.
    \begin{itemize}
        \item Aim: Find $\mu,\lambda\in\C$ such that
        \begin{equation*}
            (y'-\mu y)'-\lambda(y'-\mu y) = 0
        \end{equation*}
        \item To find the parameters, we expand the above to
        \begin{equation*}
            y''-(\mu+\lambda)y'+\mu\lambda y = 0
        \end{equation*}
        \item Comparing with the original form, we have that $a=-(\mu+\lambda)$ and $b=\mu\lambda$.
        \item It follows that $\mu,\lambda$ are the roots of $x^2+ax+b=0$, which we will call the \textbf{characteristic polynomial} of the ODE.
        \item Substitute $x=y'-\mu y$. Then we can solve
        \begin{equation*}
            x'-\lambda x = 0
        \end{equation*}
        to determine that $x=A\e[\lambda t]$.
        \item Returning the substitution, we have that
        \begin{equation*}
            y'-\mu y = A\e[\lambda t]
        \end{equation*}
        \item Since the above is of the form $y'=ay+f$, we can apply the Duhamel formula. It follows that a particular solution is
        \begin{equation*}
            A\int_0^t\e[\mu(t-\tau)]\e[\lambda\tau]\dd\tau
        \end{equation*}
        \item Thus, general solutions are of the form
        \begin{equation*}
            y(t) = B\e[\mu t]+C\e[\mu t]\int_0^t\e[(\lambda-\mu)\tau]\dd\tau
        \end{equation*}
        \item Evaluating the integral, we get
        \begin{equation*}
            y(t) = B\e[\mu t]+C\e[\mu t]\frac{\e[(\lambda-\mu)t]-1}{\lambda-\mu}
        \end{equation*}
        which simplifies (by incorporating constants, etc.) to
        \begin{equation*}
            y(t) = A_1\e[\mu t]+B_1\e[\lambda t]
        \end{equation*}
        for $\mu\neq\lambda$, or
        \begin{equation*}
            y(t) = A_1\e[\mu t]+B_1t\e[\mu t]
        \end{equation*}
        for $\mu=\lambda$.
        \item These linearly independent solutions form a basis of the space of solutions; all solutions can be expressed as a linear combination of these two functions.
    \end{itemize}
    \item If our equation is of the form $y''+ay'+by=f(t)$, then we just need to apply the Duhamel formula twice.
    \item Returning to the simple harmonic oscillator problem, we substitute $\omega=\sqrt{k/m}$ to get
    \begin{equation*}
        x'' = \omega^2x
    \end{equation*}
    \begin{itemize}
        \item The characteristic polynomial is
        \begin{equation*}
            0 = x^2+\omega^2
            = (x+i\omega)(x-i\omega)
        \end{equation*}
        \item Thus, solutions are of the form
        \begin{equation*}
            x = A_1\e[i\omega t]+B_1\e[-i\omega t]
        \end{equation*}
        \item It follows that the period is $T=2\pi/\omega$.
        \item To get a real (usable) solution, apply Euler's formula to get
        \begin{align*}
            x(t) &= A_1(\cos\omega t+i\sin\omega t)+B_1(\cos\omega t-i\sin\omega t)\\
            &= A\cos\omega t+B\sin\omega t
        \end{align*}
        where $A=A_1+B_1$, $B=iA_1-iB_1$.
        \item To match the initial condition $x(0)=x_0$, $x'(0)=v_0$, we use
        \begin{equation*}
            x(t) = x_0\cos\omega t+\frac{v_0}{\omega}\sin\omega t
        \end{equation*}
        \item In other words,
        \begin{align*}
            &
            \begin{cases}
                A=x_0\\
                \omega B=v_0
            \end{cases}
            &&
            \begin{cases}
                A_1+B_1=x_0\\
                i\omega A_1-i\omega B_1=v_0
            \end{cases}
        \end{align*}
        so
        \begin{align*}
            &
            \begin{cases}
                A=x_0\\
                B=\frac{v_0}{\omega}
            \end{cases}
            &&
            \begin{cases}
                A_1=\frac{1}{2}\left[ x_0-\frac{iv_0}{\omega} \right]\\
                B_1=\frac{1}{2}\left[ x_0+\frac{iv_0}{\omega} \right]
            \end{cases}
        \end{align*}
    \end{itemize}
\end{itemize}



\section{ODE Examples}
\begin{itemize}
    \item \marginnote{10/7:}Today, we will investigate a variety of examples of ODEs arising in real life.
    \item Michaelis-Menten kinetics: If E is an enzyme, S is its substrate, and P is the product, then the mechanism is
    \begin{equation*}
        \ce{E + S <=>[$k_1$][$k_{-1}$] ES ->[$k_2$] E + P}
    \end{equation*}
    \item The concentrations that we are concerned with are $\cnc{E},\cnc{S},\cnc{ES},\cnc{P}$.
    \item From the above mechanism, we can write the four rate laws
    \begin{align*}
        \dv{t}\cnc{S} &= -k_1\cnc{E}\cnc{S}+k_{-1}\cnc{ES}\tag{1}\\
        \dv{t}\cnc{E} &= -k_1\cnc{E}\cnc{S}+(k_{-1}+k_2)\cnc{ES}\tag{2}\\
        \dv{t}\cnc{ES} &= k_1\cnc{E}\cnc{S}-(k_{-1}+k_2)\cnc{ES}\tag{3}\\
        \dv{t}\cnc{P} &= k_2\cnc{ES}\tag{4}
    \end{align*}
    \item The initial conditions are $\cnc{S}=\cnc[0]{S}$ and $\cnc{E}=\cnc[0]{E}$.
    \item We can reduce these rate laws to the 2D system
    \begin{align*}
        \dv{t}\cnc{S} &= -k_1(\cnc[0]{E}-\cnc{ES})\cnc{S}+k_{-1}\cnc{ES}\tag{5}\\
        \dv{t}\cnc{ES} &= k_1(\cnc[0]{E}-\cnc{ES})\cnc{S}-(k_{-1}+k_2)\cnc{ES}\tag{6}
    \end{align*}
    \begin{itemize}
        \item Note that to do so, we have used two conservation laws: The conservation of the enzyme plus enzyme-substrate complex, and the conservation of the substrate plus enzyme-substrate complex plus products.
    \end{itemize}
    \item QSSA: Quasi steady-state assumption.
    \begin{itemize}
        \item Assume that $\cnc[0]{E}/\cnc[0]{S}\lll 1$.
        \item It follows that $\dv*{\cnc{ES}}{t}\approx 0$ due to saturation of the enzyme and $\cnc{S}\approx\cnc[0]{S}$ due to ever-more substrate being available.
    \end{itemize}
    \item Then
    \begin{equation*}
        \cnc{ES} = \frac{\cnc[0]{E}\cnc{S}}{K_M+\cnc{S}}
    \end{equation*}
    where $k_M=(k_{-1}+k_2)/k_1$ is the \textbf{Michaelis-Menten constant}, a usual indication of enzyme activity.
    \item Substitute the above into Equation 5:
    \begin{equation*}
        \dv{t}\cnc{S} = -\frac{v_\text{max}\cnc{S}}{k_M+\cnc{S}}
    \end{equation*}
    \begin{itemize}
        \item Note that $v_\text{max}=k_2\cnc[0]{E}$.
    \end{itemize}
    \item The above is now a differential equation of separable form; it's solution is
    \begingroup
    \allowdisplaybreaks
    \begin{align*}
        \int_{\cnc[0]{S}}^{\cnc{S}}-\frac{(k_M+z)\dd{z}}{zv_\text{max}} &= \int_0^t\dd{t}\\
        -\frac{k_M}{v_\text{max}}\log\frac{\cnc{S}}{\cnc[0]{S}}-\frac{1}{v_\text{max}}(\cnc{S}-\cnc[0]{S}) &= t\\
        \log\frac{\cnc{S}}{\cnc[0]{S}}+\frac{\cnc{S}}{k_M} &= \frac{\cnc[0]{S}-v_\text{max}t}{k_M}\\
        \frac{\cnc{S}}{\cnc[0]{S}}\e[\cnc{S}/k_M] &= \exp(\frac{\cnc[0]{S}-v_\text{max}t}{k_M})\\
        \frac{\cnc{S}}{k_M}\e[\cnc{S}/k_M] &= \frac{\cnc[0]{S}}{k_M}\exp(\frac{\cnc[0]{S}-v_\text{max}t}{k_M})\\
        \frac{\cnc{S}}{k_M} &= W\left[ \frac{\cnc[0]{S}}{k_M}\exp(\frac{\cnc[0]{S}-v_\text{max}t}{k_M}) \right]\\
        \cnc{S} &= k_MW\left[ \frac{\cnc[0]{S}}{k_M}\exp(\frac{\cnc[0]{S}-v_\text{max}t}{k_M}) \right]
    \end{align*}
    \endgroup
    \begin{itemize}
        \item Getting from line 5-6 (i.e., the introduction of $W$): Suppose we have an equation of the form $y\e[y]=x$. We cannot express $x$ in terms of $y$ using elementary functions, so we must define $W$ such that $y=W(x)$. Explicitly, $W$ is the unique function of $x$ that satisfies $W(x)\e[W(x)]=x$.
    \end{itemize}
    \item Harmonic oscillator.
    \item Recall that
    \begin{equation*}
        x''+\frac{k}{m}x = 0
    \end{equation*}
    \item Substituting $\omega=\sqrt{k/m}$, we can solve the above for
    \begin{equation*}
        x(t) = x(0)\cos(\omega t)+\frac{x'(0)}{\omega}\sin(\omega t)
    \end{equation*}
    \item This is an integrable system with $n$ degrees of freedom and $n-1$ scalar conservation laws??
    \item Conservation of mechanical energy:
    \begin{equation*}
        E = \frac{1}{2}m|x'|^2+\frac{1}{2}kx^2
    \end{equation*}
    \begin{figure}[h!]
        \centering
        \begin{tikzpicture}
            \footnotesize
            \draw
                (-3,0) -- (3,0) node[right]{$x$}
                (0,-2) -- (0,2) node[above]{$x'$}
            ;

            \draw [rex,thick]
                ellipse (6mm and 4mm)
                ellipse (12mm and 8mm)
                ellipse (18mm and 12mm)
            ;
        \end{tikzpicture}
        \caption{Conservation of mechanical energy in the harmonic oscillator.}
        \label{fig:harmonicEConservation}
    \end{figure}
    \begin{itemize}
        \item Differentiating wrt. $x$ yields
        \begin{align*}
            0 &= mx'x''+kxx'\\
            &= \dv{t}(\frac{1}{2}m(x')^2)+\dv{t}(\frac{1}{2}kx^2)
        \end{align*}
        \item This means that the solution is an ellipse in the $xx'$-plane, where each ellipse corresponds to an initial displacement and velocity.
    \end{itemize}
    \item Mathematical pendulum.
    \item Equation of motion:
    \begin{align*}
        0 &= l\theta''+g\sin\theta\\
        &= \ell\theta''\theta'+g\sin\theta\cdot\theta'\\
        &= \dv{t}\bigg( \underbrace{\frac{\ell}{2}|\theta'|^2-g\cos\theta}_E \bigg)
    \end{align*}
    \item Initial values:
    \begin{align*}
        \theta(0) &= \theta_0&
        \theta'(0) &= 0
    \end{align*}
    \item It follows from the above that
    \begin{align*}
        \frac{\ell}{2}|\theta'|^2-g\cos\theta_0 &= -g\cos\theta\\
        \dv{\theta}{t} &= \sqrt{\frac{2g}{\ell}(\cos\theta_0-\cos\theta)}\\
        \int_{\theta_0}^\theta\sqrt{\frac{\ell}{2g(\cos\theta_0-\cos\phi)}}\dd\phi &= t
    \end{align*}
    \begin{itemize}
        \item This is an elliptical integral (and thus cannot be expressed in terms of the elementary functions).
    \end{itemize}
    \item Suppose $\theta_0$ is small. Then $\theta$ is small, and we can invoke the small angle approximation $\sin\theta\approx\theta$.
    \begin{itemize}
        \item This yields an approximate equation of motion:
        \begin{equation*}
            \ell\theta''+g\theta = 0
        \end{equation*}
        \item From here, we can determine that $\theta(t)\approx\theta_0\cos\sqrt{g/\ell}\cdot t$ and $T=2\pi\sqrt{\ell/g}$.
    \end{itemize}
    \item Kepler problem.
    \item Two bodies of mass $m_1,m_2$ are located at positions $x_1,x_2$ pulling on each other gravitationally.
    \begin{itemize}
        \item The force of attraction is a conservative central force.
        \item The potential between the two masses is a function of their distance, i.e, $U(|x_1-x_2|)$.
    \end{itemize}
    \item From Newton's second and third law, we get
    \begin{align*}
        m_1x_1'' &= U'(|x_1-x_2|)\frac{x_2-x_1}{|x_2-x_1|}&
        m_2x_2'' &= U'(|x_1-x_2|)\frac{x_1-x_2}{|x_1-x_2|}
    \end{align*}
    \begin{itemize}
        \item The derivative of potential is force.
        \item The vector term provides direction.
    \end{itemize}
    \item Conservation of momentum:
    \begin{align*}
        (m_1x_1+m_2x_2)'' &= 0\\
        m_1x_1'+m_2x_2' &= C
    \end{align*}
    \begin{itemize}
        \item Let $M=m_1+m_2$. Then the center of mass
        \begin{equation*}
            \frac{m_1}{M}x_1+\frac{m_2}{M}x_2
        \end{equation*}
        moves inertially (i.e., does not accelerate or decelerate; is a stable reference frame) --- we'll define it to be the origin.
    \end{itemize}
    \item Conservation of angular momentum:
    \begin{equation*}
        [m(x_1-x_2)'\times(x_1-x_2)]' = 0
    \end{equation*}
    \begin{itemize}
        \item $m=m_1m_2/(m_1+m_2)$.
        \item $\times$ indicates the cross product.
        \item $L=m(x_1-x_2)'\times(x_1-x_2)$.
    \end{itemize}
    \item It follows that $x_1-x_2$ is always in a fixed plane, which we may call the \textbf{horizon plane}.
    \item Conservation of mechanical energy:
    \begin{align*}
        mq''+U'(|q|)\frac{q}{|q|} &= 0\\
        \frac{m}{2}|q'|^2+U(|q|) &= E
    \end{align*}
    \begin{itemize}
        \item $q=x_1-x_2$.
    \end{itemize}
    \item Introduce polar coordinates $(r,\phi)$.
    \begin{itemize}
        \item Then $r^2\phi'=\ell_0$, $r=r(\phi)$, and $\dv*{r}{\phi}=r'(t)/\phi'(t)$.
        \item It follows that
        \begin{equation*}
            \frac{m}{2}(|r'|^2+|\phi'|^2)+U(r) = E
        \end{equation*}
        \item Since $U(r)=-Gm_1m_2/r$ for Newtonian gravity,
        \begin{equation*}
            \left( \dv{r}{\phi} \right)^2+r^2 = \frac{2GMr^3}{\ell_0^2}+\frac{2Er^4}{m\ell_0^2}
        \end{equation*}
        \item The substitution $\mu=1/r$ yields
        \begin{equation*}
            \left( \dv{\mu}{\phi} \right)^2+\mu^2 = \frac{2GM}{\ell_0^2}\mu+\frac{2E}{m\ell_0^2}
        \end{equation*}
        \item Differentiating again gives
        \begin{equation*}
            2\dv{\mu}{\phi}\dv[2]{\mu}{\phi}+2\mu\dv{\mu}{\phi} = \frac{2GM}{\ell_0^2}\dv{\mu}{\phi}
        \end{equation*}
        \item Substituting $\mu=\cos(t)$ gives
        \begin{equation*}
            \dv[2]{\mu}{\phi}+\mu-\frac{GM}{\ell_0^2} = 0
        \end{equation*}
        or
        \begin{equation*}
            r = \frac{1}{GM/\ell_0^2+\varepsilon\cos(\phi-\phi_0)}
        \end{equation*}
        \begin{itemize}
            \item This is a conic section!
        \end{itemize}
    \end{itemize}
    \item Thus, for example, we can calculate the precession of Mercury.
    \item Note that while we have determined the trajectory of our 2 bodies in terms of elementary functions, the $n$-body problem cannot be solved analytically.
\end{itemize}




\end{document}