\documentclass[../psets.tex]{subfiles}

\pagestyle{main}
\renewcommand{\leftmark}{Problem Set \thesection}
\setenumerate[1]{label={\textbf{\arabic*.}}}
\setenumerate[2]{label={(\arabic*)}}

\begin{document}




\section{IVP Examples and Physical Problems}
\subsection*{Required Problems}
\begin{enumerate}
    \item \marginnote{10/12:}Classify the following ordinary differential equations (systems) by indicating the order, if they are linear, and if they are autonomous.
    \begin{enumerate}
        \item $y'(x)+y(x)=0$.
        \begin{proof}[Answer]
            {\color{white}hi}
            \begin{center}
                \small
                \renewcommand{\arraystretch}{1.2}
                \begin{tabular}{c|c|c}
                    Order & Linear? & Autonomous?\\
                    \hline
                    1 & Yes & Yes\\
                \end{tabular}
            \end{center}
        \end{proof}
        \item $y''(t)=t\sin(y(t))$.
        \begin{proof}[Answer]
            {\color{white}hi}
            \begin{center}
                \small
                \renewcommand{\arraystretch}{1.2}
                \begin{tabular}{c|c|c}
                    Order & Linear? & Autonomous?\\
                    \hline
                    2 & No & No\\
                \end{tabular}
            \end{center}
        \end{proof}
        \item $x'=-y$, $y'=2x$.
        \begin{proof}[Answer]
            {\color{white}hi}
            \begin{center}
                \small
                \renewcommand{\arraystretch}{1.2}
                \begin{tabular}{c|c|c}
                    Order & Linear? & Autonomous?\\
                    \hline
                    1 & Yes & Yes\\
                \end{tabular}
            \end{center}
        \end{proof}
        \item $y'(t)=y(t)\sin(t)+\cos(y(t))$.
        \begin{proof}[Answer]
            {\color{white}hi}
            \begin{center}
                \small
                \renewcommand{\arraystretch}{1.2}
                \begin{tabular}{c|c|c}
                    Order & Linear? & Autonomous?\\
                    \hline
                    1 & No & No\\
                \end{tabular}
            \end{center}
        \end{proof}
    \end{enumerate}
    \item Transform the following differential equations to first-order systems.
    \begin{enumerate}
        \item $y^{(3)}+2y''-y'+y=0$.
        \begin{proof}
            Let
            \begin{equation*}
                x =
                \begin{pmatrix}
                    y\\
                    y'\\
                    y''\\
                \end{pmatrix}
            \end{equation*}
            Then
            \begin{equation*}
                x' =
                \begin{pmatrix}
                    y'\\
                    y''\\
                    y^{(3)}\\
                \end{pmatrix}
            \end{equation*}
            so, by comparing components between the above two vectors and then using the original linear equation to define the last entry (with substitutions), we obtain
            \begin{empheq}[box=\fbox]{align*}
                x_1' &= x_2\\
                x_2' &= x_3\\
                x_3' &= -2x_3+x_2-x_1
            \end{empheq}
        \end{proof}
        \item $x''-t\sin x'=x$.
        \begin{proof}
            In an analogous manner to the above, we can determine that
            \begin{empheq}[box=\fbox]{align*}
                y_1' &= y_2\\
                y_2' &= y_1+t\sin y_2
            \end{empheq}
        \end{proof}
    \end{enumerate}
    \item Solve the following differential equations with initial value $x(0)=x_0$. Also identify the set of $x_0$ for which these solutions are extendable to the whole of $t\geq 0$. When a solution cannot be extended to the whole of $t\geq 0$, determine its lifespan in terms of $x_0$.\par
    \emph{Example}: Solve $x'=x^2$ with $x(0)=x_0$. By separation of variables, the solution reads
    \begin{equation*}
        \int_{x_0}^x\frac{\dd{w}}{w^2} = \int_0^t\dd\tau
    \end{equation*}
    where the integral on the left-hand side cannot pass through $w=0$. The result is
    \begin{equation*}
        -\frac{1}{x}+\frac{1}{x_0} = t
        \quad\Longleftrightarrow\quad
        x(t) = \frac{x_0}{1-x_0t}
    \end{equation*}
    When $x_0\leq 0$, the solution exists throughout $t\geq 0$. When $x_0>0$, the solution only exists in $[0,1/x_0)$\footnote{This is an interval of existence. The problem asks for the lifespan. Thus, this example is wrong. It is probably what led most people to give an interval of existence in their answer instead of a lifespan.}.
    \begin{enumerate}
        \item $x'=x\sin t$.
        \begin{proof}
            By separation of variables, the solution reads
            \begin{equation*}
                \int_{x_0}^x\frac{\dd{w}}{w} = \int_0^t\sin\tau\dd\tau
            \end{equation*}
            The result is
            \begin{equation*}
                \ln\frac{x}{x_0} = 1-\cos t
                \quad\Longleftrightarrow\quad
                \boxed{x(t) = x_0\e[1-\cos t]}
            \end{equation*}
            The set of $x_0$ for which this solution is extendable to the whole of $t\geq 0$ is $\boxed{\R}$.
        \end{proof}
        \item $x'=t^2\tan x$.
        \begin{proof}
            By separation of variables, the solution reads
            \begin{equation*}
                \int_{x_0}^x\cot w\dd{w} = \int_0^t\tau^2\dd\tau
            \end{equation*}
            where the integral on the left-hand side cannot pass through $x=\pi n$ for any $n\in\Z$. The result is
            \begin{equation*}
                \ln\left| \frac{\sin x}{\sin x_0} \right| = \frac{t^3}{3}
                \quad\Longleftrightarrow\quad
                \boxed{x(t) = \arcsin(\e[t^3/3]\sin x_0)}
            \end{equation*}
            Based on the above, the solution is extendable to the whole of $t\geq 0$ only when $\e[t^3/3]\sin x_0$ remains in the domain of arcsine, i.e., $[-1,1]$ for all $t$. Based on what we know about exponential functions, this happens iff $\sin x_0=0$, i.e., iff
            \begin{equation*}
                \boxed{
                    x_0 = \pi n
                    ,\quad
                    n\in\Z
                }
            \end{equation*}
            When $x_0\neq\pi n$ for some $n\in\Z$, the lifespan is
            \begin{equation*}
                \boxed{\sqrt[3]{3\ln\left| \frac{1}{\sin(x_0)} \right|}}
            \end{equation*}
        \end{proof}
        \item $x'=1+x^2$.
        \begin{proof}
            By separation of variables, the solution reads
            \begin{equation*}
                \int_{x_0}^x\frac{1}{1+w^2}\dd{w} = \int_0^t\dd\tau
            \end{equation*}
            The result is
            \begin{equation*}
                \arctan(x)-\arctan(x_0) = t
                \quad\Longleftrightarrow\quad
                \boxed{x(t) = \tan(t+\arctan(x_0))}
            \end{equation*}
            Since the tangent function blows up periodically, the set of $x_0$ for which the solution is extendable to the whole of $t\geq 0$ is $\boxed{\emptyset}$. With respect to lifespan, we use a similar approach to the above. In particular, the solution exists for $t\geq 0$ such that $t+\arctan(x_0)\in(-\pi/2,\pi/2)$, i.e., for $t$ in $[0,\pi/2-\arctan(x_0))$. Therefore, the lifespan is
            \begin{equation*}
                \boxed{\frac{\pi}{2}-\arctan(x_0)}
            \end{equation*}
        \end{proof}
        \item $x'=\e[x]\sin t$.
        \begin{proof}
            By separation of variables, the solution reads
            \begin{equation*}
                \int_{x_0}^x\e[-w]\dd{w} = \int_0^t\sin\tau\dd\tau
            \end{equation*}
            The result is
            \begin{equation*}
                -\e[-x]+\e[-x_0] = 1-\cos t
                \quad\Longleftrightarrow\quad
                \boxed{x(t) = -\ln(\e[-x_0]-1+\cos t)}
            \end{equation*}
            The set of $x_0$ for which the solution is extendable to the whole of $t\geq 0$ is
            \begin{equation*}
                \boxed{\{x_0\in\R\mid x_0<\ln(1/2)\}}
            \end{equation*}
            When $x_0\geq\ln(1/2)$, the lifespan is
            \begin{equation*}
                \boxed{\arccos(1-\e[-x_0])}
            \end{equation*}
        \end{proof}
    \end{enumerate}
    \item Consider the harmonic oscillator equation, as mentioned in class:
    \begin{equation*}
        x''+\mu x'+\omega^2x = 0
    \end{equation*}
    Here, the initial data $x(0)=x_0$ and $x'(0)=x_1$ are real numbers.
    \begin{enumerate}
        \item Derive two linearly independent \emph{real} solutions when $\mu>0$. (Hint: You should consider the cases $\mu<2\omega$ and $\mu>2\omega$ separately.)
        \begin{proof}
            From class, we know that the form of the two linearly independent solutions to such an equation depends on whether or not the roots of the characteristic polynomial
            \begin{equation*}
                r^2+\mu r+\omega^2 = 0
            \end{equation*}
            are equal. At this point, we will take the hint and divide into three cases $(\mu>|2\omega|$, $\mu=2\omega$, and $\mu<|2\omega|$).\par
            \underline{$\mu>|2\omega|$}: In this case, the two roots
            \begin{align*}
                r_1 &= \frac{-\mu+\sqrt{\mu^2-4\omega^2}}{2}&
                r_2 &= \frac{-\mu-\sqrt{\mu^2-4\omega^2}}{2}
            \end{align*}
            are not equal. Thus, the two linearly independent solutions are
            \begin{equation*}
                \boxed{\e[r_1t],\e[r_2t]}
            \end{equation*}
            Moreover, since $r_1,r_2$ are both real, the above constitute two linearly independent \emph{real} solutions, as desired.\par 
            \underline{$\mu=2\omega$}: In this case, the two roots are equal, i.e., $r_1=r_2$. Specifically, both roots coincide at $-\mu/2$. Thus, the two linearly independent solutions are
            \begin{equation*}
                \boxed{\e[-\mu t/2],t\e[-\mu t/2]}
            \end{equation*}
            For the same reason as in the previous case, the above are also real solutions, as desired.\par
            \underline{$\mu<|2\omega|$}: In this case, the two roots are once again unequal. Thus, the two linearly independent solutions are
            \begin{equation*}
                \e[r_1t],\e[r_2t]
            \end{equation*}
            However, this time, $r_1,r_2$ are \emph{not} real, so the above solutions are not real either. However, we can still obtain real solutions from these using Euler's formula. Notice that
            \begin{align*}
                r_1 &= -\frac{\mu}{2}+i\sqrt{\omega^2-\frac{\mu^2}{4}}&
                r_2 &= -\frac{\mu}{2}-i\sqrt{\omega^2-\frac{\mu^2}{4}}&
            \end{align*}
            where both $-\mu/2$ and $\sqrt{\omega^2-\mu^2/4}$ are real numbers, which we may call $\alpha$ and $\beta$, respectively. Thus, since any linear combination of $\e[r_1t],\e[r_2t]$ is another solution, we know in particular that the functions
            \begin{align*}
                \frac{1}{2}(\e[r_1t]+\e[r_2t]) &= \frac{1}{2}(\e[\alpha t+it\beta]+\e[\alpha t-it\beta])\\
                &= \frac{1}{2}\e[\alpha t][(\cos(t\beta)+i\sin(t\beta))+(\cos(t\beta)-i\sin(t\beta))]\\
                &= \e[\alpha t]\cos(t\beta)\\
                &= \e[-\mu t/2]\cos(t\sqrt{\omega^2-\frac{\mu^2}{4}})
            \end{align*}
            and
            \begin{align*}
                \frac{1}{2i}(\e[r_1t]-\e[r_2t]) &= \frac{1}{2i}(\e[\alpha t+it\beta]-\e[\alpha t-it\beta])\\
                &= \frac{1}{2i}\e[\alpha t][(\cos(t\beta)+i\sin(t\beta))-(\cos(t\beta)-i\sin(t\beta))]\\
                &= \e[\alpha t]\sin(t\beta)\\
                &= \e[-\mu t/2]\sin(t\sqrt{\omega^2-\frac{\mu^2}{4}})
            \end{align*}
            are solutions. These solutions are linearly independent. Additionally, they are real. Therefore, we may take them, restated as
            \begin{equation*}
                \boxed{\e[-\mu t/2]\cos(t\sqrt{\omega^2-\frac{\mu^2}{4}}),\e[-\mu t/2]\sin(t\sqrt{\omega^2-\frac{\mu^2}{4}})}
            \end{equation*}
            to be our desired linearly independent real solutions in this case.
        \end{proof}
        \item Recall that $\mu=b/m$ and $\omega^2=k/m$. Recall also that the mechanical energy for the oscillator reads
        \begin{equation*}
            E = \frac{1}{2}m|x'|^2+\frac{1}{2}kx^2
        \end{equation*}
        Compute the time derivative of $E$ and conclude that $E$ is exponentially decaying for $b>0$, i.e., the mechanical energy is not conserved in this case. Does this violate the law of conservation of mechanical energy?
        \begin{proof}
            % From part (a), we have that
            % \begin{align*}
            %     x(t) &= A\e[c_1t]+B\e[c_2t]&
            %     x'(t) &= Ac_1\e[c_1t]+Bc_2\e[c_2t]
            % \end{align*}
            % Thus,
            % \begin{equation*}
            %     E = \frac{1}{2}m\left| Ac_1\e[c_1t]+Bc_2\e[c_2t] \right|^2+\frac{1}{2}k\left( A\e[c_1t]+B\e[c_2t] \right)^2
            % \end{equation*}
            % Applying the chain rule, we have that
            % \begin{align*}
            %     \dv{E}{t} &= m\left( Ac_1\e[c_1t]+Bc_2\e[c_2t] \right)\left( Ac_1^2\e[c_1t]+Bc_2^2\e[c_2t] \right)+k\left( A\e[c_1t]+B\e[c_2t] \right)\left( Ac_1\e[c_1t]+Bc_2\e[c_2t] \right)\\
            %     &= mx'x''+kxx'
            % \end{align*}


            Applying the chain rule, we have that
            \begin{equation*}
                \dv{E}{t} = mx'x''+kxx'
            \end{equation*}
            It follows that
            \begin{align*}
                \dv{E}{t} &= mx'(-\mu x'-\omega^2x)+kxx'\\
                &= x'(-bx'-kx)+kxx'\\
                &= -b(x')^2
            \end{align*}
            Now $x'\neq 0$ (as an exponential function). Hence, $(x')^2>0$. This and $b>0$ show that $\dv{E}{t}$ is always equal to a negative value. But this is characteristic of exponential decay, as desired.\par
            Mechanical energy is conserved; it is dispersed from system to surroundings by the drag $b$.
        \end{proof}
    \end{enumerate}
    \item Use the transformation $y=tw$ to convert
    \begin{equation*}
        y' = f(y/t)
    \end{equation*}
    to an ODE in $w$. Write down this equation for $w$. Use this transformation to solve
    \begin{equation*}
        tyy'+4t^2+y^2 = 0
        ,\quad
        y(2) = -7
    \end{equation*}
    Determine the lifespan (you can use a calculator for an approximate value).
    \begin{proof}
        If $y=tw$, then
        \begin{equation*}
            \dv{y}{t} = w+t\dv{w}{t}
        \end{equation*}
        Thus, the ODE in terms of $w$ is
        \begin{equation*}
            \boxed{\dv{w}{t} = \frac{f(w)-w}{t}}
        \end{equation*}
        which is a separable differential equation.\par
        We have that
        \begin{equation*}
            tyy'+4t^2+y^2 = 0
            \quad\Longleftrightarrow\quad
            y' = -4\left( \frac{y}{t} \right)^{-1}-\frac{y}{t}
        \end{equation*}
        Using the above transformation yields
        \begin{equation*}
            \dv{w}{t} = \frac{(-4w^{-1}-w)-w}{t}
        \end{equation*}
        Transforming the initial condition as well gives
        \begin{equation*}
            w(2) = \frac{y(2)}{2} = -\frac{7}{2}
        \end{equation*}
        We can simplify and solve the above as follows.
        \begin{align*}
            \frac{\dd{w}}{-4w^{-1}-2w} &= \frac{\dd{t}}{t}\\
            -\frac{1}{4}\int_{-7/2}^w\frac{2v\dd{v}}{v^2+2} &= \int_2^t\frac{\dd\tau}{\tau}\\
            -\frac{1}{4}\left[ \ln(w^2+2)-\ln(14.25) \right] &= \ln(\frac{t}{2})\\
            w &= \pm\frac{1}{t^2}\sqrt{228-2t^4}\\
            \Aboxed{y(t) &= -\frac{1}{t}\sqrt{228-2t^4}}
        \end{align*}
        Note that we pick the negative in the final step to fit the initial condition.\par
        The initial value of $t$ is 2 by hypothesis. The final value of $t$ can be determined by calculating when $228-2t^4=0$. This occurs such that the interval of existence is approximately $[2,3.27]$ and the lifespan is thus approximately $\boxed{1.27}$.
    \end{proof}
    \item Use the transformation $w=y^{1-\alpha}$ to convert Bernoulli's equation
    \begin{equation*}
        y'+p(t)y = q(t)y^\alpha
        ,\quad
        \alpha\neq 0,1
    \end{equation*}
    to an ODE in $w$. Write down this equation for $w$. Use this transformation to solve
    \begin{equation*}
        6y'-2y = ty^4
        ,\quad
        y(0) = -2
    \end{equation*}
    Determine the lifespan (you can use a calculator for an approximate value).
    \begin{proof}
        If $w=y^{1-\alpha}$, then
        \begin{align*}
            y &= w^{1/(1-\alpha)}&
            \dv{y}{t} &= \frac{w^{\alpha/(1-\alpha)}}{1-\alpha}\dv{w}{t}
        \end{align*}
        Thus, the ODE in terms of $w$ is
        \begin{equation*}
            \boxed{\frac{w^{\alpha/(1-\alpha)}}{1-\alpha}\dv{w}{t}+p(t)w^{1/(1-\alpha)} = q(t)w^{\alpha/(1-\alpha)}}
        \end{equation*}
        which is an exact differential equation.\par
        We have that
        \begin{equation*}
            6y'-2y = ty^4
            \quad\Longleftrightarrow\quad
            y'+\left( -\frac{1}{3} \right)y = \left( \frac{t}{6} \right)y^4
        \end{equation*}
        Using the above transformation yields
        \begin{equation*}
            -\frac{w^{-4/3}}{3}\dv{w}{t}-\frac{w^{-1/3}}{3} = \frac{tw^{-4/3}}{6}
        \end{equation*}
        We can simplify and evaluate the above as follows.
        \begin{align*}
            \frac{1}{3}w^{-4/3}\dv{w}{t}+\frac{1}{3}w^{-1/3} &= -\frac{t}{6}w^{-4/3}\\
            \dv{w}{t}+w &= -\frac{t}{2}\\
            \e[t]\dv{w}{t}+\e[t]w &= -\frac{t}{2}\e[t]\\
            \dv{t}(\e[t]w) &= -\frac{t}{2}\e[t]\\
            \e[t]w &= -\frac{1}{2}\int t\e[t]\dd{t}\\
            &= -\frac{1}{2}\e[t](t-1)+C\\
            w &= -\frac{1}{2}(t-1)+C\e[-t]\\
            y^{-3} &= -\frac{1}{2}(t-1)+C\e[-t]\\
            y &= \left[ -\frac{1}{2}(t-1)+C\e[-t] \right]^{-1/3}
        \end{align*}
        We now apply the initial condition.
        \begin{align*}
            \left[ -\frac{1}{2}(0-1)+C\e[-0] \right]^{-1/3} &= y(0)\\
            \left[ \frac{1}{2}+C \right]^{-1/3} &= -2\\
            C &= -\frac{5}{8}
        \end{align*}
        Therefore, the solution to the ODE in question is
        \begin{equation*}
            \boxed{y(t) = \left[ -\frac{1}{2}(t-1)-\frac{5}{8}\e[-t] \right]^{-1/3}}
        \end{equation*}
        The equation \fbox{does not have finite lifespan}.
    \end{proof}
    \item Show that
    \begin{equation*}
        (4bxy+3x+5)y'+3x^2+8ax+2by^2+3y = 0
    \end{equation*}
    is an exact equation, no matter what value $a,b$ take. Find the implicit relation satisfied by the solution $y(x)$ and $x$.
    \begin{proof}
        To show that an equation of the form $g\dv*{y}{x}+f=0$ is exact, it will suffice to confirm that
        \begin{equation*}
            \pdv{g}{x} = \pdv{f}{y}
        \end{equation*}
        Since the equation in question is of this form, we may evaluate directly:
        \begin{align*}
            \pdv{g}{x} &= 4by+3&
            \pdv{f}{y} &= 4by+3
        \end{align*}
        By transitivity, we have the desired result.\par
        Having confirmed that the given equation is exact as written (i.e., we do \emph{not} need integrating factors to make it so), we now want to find $F$ such that $\pdv*{F}{x}=f$ and $\pdv*{F}{y}=g$. Starting with the former constraint, we can determine that
        \begin{align*}
            F(x,y) &= \int(3x^2+8ax+2by^2+3y)\dd{x}\\
            &= x^3+4ax^2+2bxy^2+3xy+h(y)
        \end{align*}
        where $h(y)$ is a function of integration (as opposed to a constant of integration). We now differentiate with respect to $y$.
        \begin{equation*}
            \pdv{F}{y} = 4bxy+3x+\dv{h}{y}
        \end{equation*}
        Knowing that $\pdv*{F}{y}=g$, we can use the above equation to solve for $h$ as follows.
        \begin{align*}
            4bxy+3x+5 &= 4bxy+3x+\dv{h}{y}\\
            \dv{h}{y} &= 5\\
            h(y) &= 5y
        \end{align*}
        Therefore, we know that
        \begin{equation*}
            F(x,y) = x^3+4ax^2+2bxy^2+3xy+5y
        \end{equation*}
        Since $F$ is constant along any solution by construction, the desired implicit relation is given by setting the above equal to some constant $C\in\R$, as follows.
        \begin{equation*}
            \boxed{x^3+4ax^2+2bxy^2+3xy+5y = C}
        \end{equation*}
    \end{proof}
    \item Let $a,b$ be constants. For Euler's equation
    \begin{equation*}
        t^2y''+aty'+by = f(t)
    \end{equation*}
    consider the transformation $w(\tau)=y(\e[\tau])$. What is the differential equation satisfied by $w(\tau)$? Use this transformation to solve
    \begin{equation*}
        2t^2y''+3ty'-15y=0
        ,\quad
        y(1) = 0
        ,\quad
        y'(1) = 1
    \end{equation*}
    \begin{proof}
        % We use the coordinate transform tools from Section 1.4 of \textcite{bib:Teschl}. To facilitate the comparison, we change the notation of Euler's equation and our coordinate transform to
        % \begin{align*}
        %     t^2\ddot{x}+at\dot{x}+bx &= f(t)&
        %     y(s) &= x(\e[s])
        % \end{align*}
        % We will change back before reporting our final results. At this point, we may define
        % \begin{align*}
        %     \eta(t,x) &= x&
        %     \sigma(t) &= \ln(t)&
        %     \tau(s) &= \e[s]&
        %     \xi(s,y) &= y
        % \end{align*}
        % Then we know that the differential equation satisfied by $y(s)$ is
        % \begin{align*}
        %     \dot{y} &= \dot{\tau}\left( {\pdv{\eta}{t}}(\tau,\xi)+\pdv{\eta}{x}f(\tau,\xi) \right)\\
        %     &= \e[s]\left( [0]_{(\e[s],y)}+[1]_{(\e[s],y)} \right)
        % \end{align*}
        % In the original notation,

        % takes points $(t,y)$ in the product space of the original independent and dependent variable to points $(\tau,w)$ in the product space of the new independent and dependent variable


        The given transformation can be split into the following two transformations (one for the dependent variable and one for the independent variable).
        \begin{align*}
            w &= y&
                t &= \e[\tau]\\
            &&
                \tau &= \ln(t)
        \end{align*}
        Thus, differentiating our new dependent variable with respect to our new independent variable, we get
        \begin{align*}
            \dv{w}{\tau} &= \dv{y}{t}\dv{t}{\tau}\\
            w'(\tau) &= y'(t)\e[\tau]\\
            y'(t) &= \frac{w'(\tau)}{\e[\tau]}
        \end{align*}
        Differentiating again (and substituting in the above in a later step), we get
        \begin{align*}
            \dv{\tau}(\dv{w}{\tau}) &= \dv{\tau}(\dv{y}{t})\cdot\dv{t}{\tau}+\dv{y}{t}\cdot\dv{\tau}(\dv{t}{\tau})\\
            \dv[2]{w}{\tau} &= \dv{t}(\dv{y}{t})\dv{t}{\tau}\cdot\dv{t}{\tau}+\dv{y}{t}\cdot\dv[2]{t}{\tau}\\
            &= \dv[2]{y}{t}\left( \dv{t}{\tau} \right)^2+\dv{y}{t}\dv[2]{t}{\tau}\\
            w''(\tau) &= y''(t)[\e[\tau]]^2+y'(t)\e[\tau]\\
            &= y''(t)\e[2\tau]+w'(\tau)\\
            y''(t) &= \frac{w''(\tau)-w'(\tau)}{\e[2\tau]}
        \end{align*}
        We can now substitute our transformed definitions of $y,y',y'',t$ back into the original Euler's equation.
        \begin{align*}
            t^2y''+aty'+by &= f(t)\\
            [\e[\tau]]^2\cdot\frac{w''(\tau)-w'(\tau)}{\e[2\tau]}+a\cdot\e[\tau]\cdot\frac{w'(\tau)}{\e[\tau]}+b\cdot w(\tau) &= f(\e[\tau])\\
            [w''(\tau)-w'(\tau)]+aw'(\tau)+bw(\tau) &= f(\e[\tau])\\
            \Aboxed{w''(\tau)+(a-1)w'(\tau)+bw(\tau) &= f(\e[\tau])}
        \end{align*}
        This is an inhomogeneous second-order ODE with constant coefficients; we know how to solve these.\par
        We now consider the given example. We can transform it into the general form by dividing through by 2 to yield
        \begin{equation*}
            t^2y''+\frac{3}{2}ty'-\frac{15}{2}y = 0
        \end{equation*}
        Applying the transformation, we get
        \begin{equation*}
            w''+\frac{1}{2}w'-\frac{15}{2}w = 0
        \end{equation*}
        as our new differential equation,
        \begin{equation*}
            \tau = \ln(1) = 0
        \end{equation*}
        as our new initial time, and
        \begin{align*}
            w(0) &= y(1) = 0&
            w'(0) &= y'(1)\e[0] = 1
        \end{align*}
        as our new initial conditions. We can solve the characteristic polynomial for this differential equation as follows\footnote{Refer to Theorem 3.7 in \textcite{bib:Teschl} for a better justification of the solution. Alternatively, we can transform it into a 2D planar system and solve it using JNF.}.
        \begin{align*}
            0 &= z^2+\frac{1}{2}z-\frac{15}{2}\\
            &= (z-5/2)(z+3)\\
            z &= \frac{5}{2},-3
        \end{align*}
        Therefore, the general solution is
        \begin{equation*}
            w(\tau) = A\e[5\tau/2]+B\e[-3\tau]
        \end{equation*}
        where $A,B\in\R$. We can use the initial conditions to solve for specific values of $A,B$. In particular, $A,B$ are the solutions to the system of equations
        \begin{align*}
            0 = w(0)  &= A+B&
            1 = w'(0) &= \frac{5}{2}A-3B
        \end{align*}
        or
        \begin{align*}
            A &= \frac{2}{11}&
            B &= -\frac{2}{11}
        \end{align*}
        Thus, the solution to the transformed IVP is
        \begin{equation*}
            w(\tau) = \frac{2}{11}\e[5\tau/2]-\frac{2}{11}\e[-3\tau]
        \end{equation*}
        Using our inverse coordinate transforms, we can determine that the solution to the original IVP is
        \begin{align*}
            y(t) &= w(\ln(t))\\
            &= \frac{2}{11}\e[5\ln(t)/2]-\frac{2}{11}\e[-3\ln(t)]\\
            &= \frac{2}{11}\left( \e[\ln(t)] \right)^{5/2}-\frac{2}{11}\left( \e[\ln(t)] \right)^{-3}\\
            \Aboxed{y(t) &= \frac{2}{11}t^{5/2}-\frac{2}{11}t^{-3}}
        \end{align*}
    \end{proof}
    \item Suppose there is a capacitor with capacitance $C$ being charged by a battery of fixed voltage $V_0$. Suppose there is a resistor $R$ connected to $C$. Then the charge $Q(t)$ of the capacitor satisfies the differential equation
    \begin{equation*}
        RQ'(t)+\frac{Q(t)}{C} = V_0
    \end{equation*}
    This is the equation for an RC charging circuit.\par
    Find the explicit solution of this equation with $Q(0)=0$. Explain why the product $RC$ is important in determining the charging time. For $R=\SI{e3}{\ohm}$, $V_0=\SI{1}{\volt}$, $C=\SI{1}{\micro\farad}$, how much time does it take for the capacitor to be charged to 98\%? (You may use a calculator.)
    \begin{proof}
        We can evaluate the ODE as follows.
        \begin{align*}
            \dv{Q}{t}+\frac{1}{RC}Q &= V_0\\
            \e[t/RC]\dv{Q}{t}+\frac{1}{RC}\e[t/RC]Q &= \e[t/RC]V_0\\
            \dv{t}(Q\e[t/RC]) &= \e[t/RC]V_0\\
            Q\e[t/RC] &= RCV_0\e[t/RC]+C_1\\
            Q(t) &= RCV_0+C_1\e[-t/RC]
        \end{align*}
        We now apply the initial condition.
        \begin{align*}
            0 &= Q(0)\\
            &= RCV_0+C_1\\
            C_1 &= -RCV_0
        \end{align*}
        Therefore, the solution to the ODE in question is
        \begin{equation*}
            \boxed{Q(t) = RCV_0\left( 1-\e[-t/RC] \right)}
        \end{equation*}
        The product $RC$ (technically referred to as the time constant) is important in determining charging time because it is directly proportional to the rate of exponential charging. Indeed, if $RC$ doubles, the capacitor will take twice as long to charge (and vice versa, for example, if $RC$ halves).\par
        The amount of time it takes for the capacitor to charge to 98\% under the given conditions ($R=\SI{e3}{\ohm}$ and $C=\SI{e-6}{\farad}$) may be determined as follows.
        \begin{align*}
            0.98 &= 1-\e[-t/RC]\\
            t &= -RC\ln(0.02)\\
            \Aboxed{t &= \SI{3.9e-3}{\second}}
        \end{align*}
    \end{proof}
    \item A parachutist is falling from a plane. Suppose the parachute is opened at height $H$, when the falling velocity is $v_0$. Suppose that the air resistance exerted on the parachute is proportional to the square of the velocity with ratio $\eta$. Let the gravitational constant be $g$, and suppose that the total mass of the parachutist and the parachute is $m$. Write down the differential equation satisfied by the shift $x$, together with the initial conditions. Solve this IVP. What is the velocity as $t\to +\infty$? Can you derive the final velocity based on physical considerations?
    \begin{proof}
        For the sake of simplicity, we will write a one-dimensional differential equation corresponding to vertical displacement. Let's begin.\par
        When the parachutist is falling freely, there is only one (idealized) force acting on them: gravity ($F_g$). As soon as the parachute is opened, another force is added to the mix: drag ($F_d$). By Newton's second law, the net force is equal to the parachutist/parachute's mass times their acceleration. Taking a convention of upwards displacement being positive, we can thus write that
        \begin{equation*}
            \sum F_z = F_d-F_g = ma
        \end{equation*}
        Since $a=x''$, $F_g=mg$, and $F_d=\eta v^2=\eta(x')^2$, the differential equation satisfied by the shift $x$ is
        \begin{equation*}
            mx'' = \eta(x')^2-mg
        \end{equation*}
        Let the time at which the parachute is opened be $t=0$. Then the initial conditions are
        \begin{align*}
            x(0) &= H&
            x'(0) &= v_0
        \end{align*}
        To solve this IVP, we substitute $v=x'$ and evaluate the resulting first-order differential equation to start:
        \begin{align*}
            mv' &= \eta v^2-mg\\
            \frac{\dd{v}}{v^2-mg/\eta} &= \frac{\eta}{m}\dd{t}\\
            \int_{v_0}^v\frac{\dd{w}}{w^2-mg/\eta} &= \int_0^t\frac{\eta}{m}\dd{\tau}\\
            \sqrt{\frac{\eta}{mg}}\left[ \coth^{-1}\left( \sqrt{\frac{\eta}{mg}}v_0 \right)-\coth^{-1}\left( \sqrt{\frac{\eta}{mg}}v \right) \right] &= \frac{\eta}{m}t\\
            v(t) &= \sqrt{\frac{mg}{\eta}}\coth(\coth^{-1}\left( \sqrt{\frac{\eta}{mg}}v_0 \right)-\sqrt{\frac{g\eta}{m}}t)
        \end{align*}
        Returning the substitution $v=x'$, we can determine that
        \begin{align*}
            x' &= \sqrt{\frac{mg}{\eta}}\coth(\coth^{-1}\left( \sqrt{\frac{\eta}{mg}}v_0 \right)-\sqrt{\frac{g\eta}{m}}t)\\
            \int_H^x\dd\xi &= \int_0^t\sqrt{\frac{mg}{\eta}}\coth(\coth^{-1}\left( \sqrt{\frac{\eta}{mg}}v_0 \right)-\sqrt{\frac{g\eta}{m}}\tau)\dd\tau\\
            x-H &= \left[ -\frac{m}{\eta}\ln(\sinh(\coth^{-1}\left( \sqrt{\frac{\eta}{mg}}v_0 \right)-\sqrt{\frac{g\eta}{m}}\tau)) \right]_0^t\\
            \Aboxed{x(t) &= H-\frac{m}{\eta}\ln(\sinh(\coth^{-1}\left( \sqrt{\frac{\eta}{mg}}v_0 \right)-\sqrt{\frac{g\eta}{m}}t))+\frac{m}{\eta}\ln(\frac{1}{v_0\cdot\sqrt{\frac{\eta}{mg}-\frac{1}{v_0^2}}})}
        \end{align*}
        Since $\coth(t)\to -1$ as $t\to -\infty$, the final velocity approaches
        \begin{equation*}
            \boxed{v_\infty = -\sqrt{\frac{mg}{\eta}}}
        \end{equation*}
        Note that we can also solve for $v_\infty$ using physical considerations; at $t=\infty$, the drag and gravitational forces will balance, i.e., we will have
        \begin{align*}
            F_d &= F_g\\
            \eta v_\infty^2 &= mg\\
            v_\infty &= \pm\sqrt{\frac{mg}{\eta}}\\
            &= -\sqrt{\frac{mg}{\eta}}
        \end{align*}
    \end{proof}
\end{enumerate}


\subsection*{Bonus Problems}
\begin{enumerate}
    \item \textbf{The Catenoid.} Suppose there are two metal rings of radius $a$ placed parallel to each other in an $xyz$-coordinate space, with the $x$-axis passing through their centers. Suppose these two rings are contained in the planes $x=l$ and $x=-l$, respectively. An axial symmetric soap film is spanned by these two rings. Suppose its shape is obtained by rotating the graph of the function $y=y(x)$ with respect to the $x$-axis. In order to attain a stable configuration, the surface area is supposed to be minimal among all such surfaces of revolution.
    \begin{enumerate}
        \item Write down the surface area functional in terms of $y(x)$, its derivative, and the boundary conditions for this variational problem.
        \begin{proof}
            The surface area functional is\footnote{See SOR-SA.pdf from AP BC Calculus for a derivation.}
            \begin{equation*}
                \boxed{J[y] = \int_{-l}^l2\pi y\sqrt{1+\left( \dv{y}{x} \right)^2}\dd{x}}
            \end{equation*}
        \end{proof}
        \item Derive the Euler-Lagrange equation and find the solution. The shape is called a \textbf{catenoid}.
        \begin{proof}
            As in class with the Brachistochrone problem, our functional is of the form $J[y]=\int_a^bF(x,y(x),y'(x))\dd{x}$. Thus, the derivation of the relevant Euler-Lagrange equation is symmetric, and yields
            \begin{equation*}
                {\pdv{F}{z}}(x,y(x),y'(x))-\dv{x}\left[ {\pdv{F}{w}}(x,y(x),y'(x)) \right] = 0
            \end{equation*}
            as a necessary condition for $y$ to be an extrema of $J[y]$. Again, note that in our specific functional, $F(x,z,w)=F(z,w)$ (i.e., there is no dependence on $x$), allowing us to actually solve the above equation. We can take this even farther using the same logic from class to get
            \begin{equation*}
                F-\dv{F}{w}\dv{y}{x} = A
            \end{equation*}
            as a necessary condition, where $A\in\R$ depends on the initial/boundary conditions. In particular, we may now substitute in our specific definition of $F$ and solve for $y$ as follows.
            \begin{align*}
                A &= 2\pi y\sqrt{1+(y')^2}-\left[ 2\pi y\cdot\frac{1}{2\sqrt{1+(y')^2}}\cdot 2y' \right]\cdot y'\\
                \frac{A}{2\pi y} &= \sqrt{1+(y')^2}-\frac{(y')^2}{\sqrt{1+(y')^2}}\\
                \frac{A^2}{4\pi^2y^2} &= [1+(y')^2]-\frac{2(y')^2\sqrt{1+(y')^2}}{\sqrt{1+(y')^2}}+\frac{(y')^4}{1+(y')^2}\\
                \frac{A^2}{4\pi^2y^2} &= 1-(y')^2+\frac{(y')^4}{1+(y')^2}\\
                \frac{A^2}{4\pi^2y^2}+\frac{A^2}{4\pi^2y^2}(y')^2 &= [1+(y')^2]-[(y')^2+(y')^4]+(y')^4\\
                \frac{A^2}{4\pi^2y^2}+\frac{A^2}{4\pi^2y^2}(y')^2 &= 1\\
                (y')^2 &= \frac{4\pi^2y^2}{A^2}-1\\
                \dv{y}{x} &= \frac{2\pi}{A}\sqrt{y^2-\frac{A^2}{4\pi^2}}\\
                y(x) &= \cosh(\frac{2\pi}{A}x)
            \end{align*}
            Knowing that $y(l)=y(-l)=a$, we can determine that
            \begin{align*}
                a &= \cosh(\frac{2\pi l}{A})\\
                A &= \frac{2\pi l}{\cosh^{-1}(a)}
            \end{align*}
            Therefore, our final solution is
            \begin{equation*}
                \boxed{y(x) = \cosh(\frac{\cosh^{-1}(a)}{l}x)}
            \end{equation*}
            Note that since we can easily construct surfaces of revolution with greater surface area, this is a minimum, not a maximum.
        \end{proof}
        \item If the two rings are very far away from each other, i.e., $l$ is very large, will the catenoid still be of minimal area among all competing surfaces that span these two rings? You do not have to give a mathematically rigorous answer; just imagine the physical situation. (Hint: What about two distinct disks spanned by these two rings?)
        \begin{proof}
            Imagine pulling two rings spanned by a soap film farther and farther apart. The soap film won't be stable forever; indeed, at some point, it will collapse in the middle and the two halves will retreat onto their distinct rings. Indeed, at some point, the area $2\pi a^2$ will be smaller than the ever-increasing (and non-asymptotic) surface area of the catenoid. Thus, the catenoid is a local minimum of the variational problem, not necessarily a global minimum.
        \end{proof}
    \end{enumerate}
    \item \textbf{A Formulation of the Isoperimetric Problem.} Recall from multivariable calculus that in order to find a local extremum of the function $f(x_1,\dots,x_n)$ under the constraint $g(x_1,\dots,x_n)=0$, we can introduce a parameter $\lambda$ called the \textbf{Lagrange multiplier} and find the stationary point of the function
    \begin{equation*}
        f(x_1,\dots,x_n)-\lambda g(x_1,\dots,x_n)
    \end{equation*}
    \begin{enumerate}
        \item Write down the equations that must be satisfied by the stationary point $(x_1,\dots,x_n)$ of the function $f-\lambda g$ with the parameter $\lambda$ involved.
        \begin{proof}
            Based on my notes, I should be able to solve this problem with ease. The method of Lagrange multiplies is explained in \textcite{bib:CAAGThomasNotes}.
        \end{proof}
        \item Use the Lagrange multiplier method to find the maxima and minima of $f(x,y)=x+y$ under the constraint $x^2+y^2=1$.
        \item Now let us generalize this method to functionals. If we aim to find the extrema of a functional
        \begin{equation*}
            J[y] = \int_a^bF(x,y(x),y'(x))\dd{x}
        \end{equation*}
        under the constraint
        \begin{equation*}
            R[y] = \int_a^bG(x,y(x),y'(x))\dd{x} = 0
        \end{equation*}
        where $F(x,z,w)$ and $G(x,z,w)$ are known functions, we can try to find the extrema of the functional
        \begin{equation*}
            J[y]-\lambda R[y]
        \end{equation*}
        first. What is the Euler-Lagrange equation satisfied by this extrema (with $\lambda$ involved)?
        \item Now let us consider a version of the isoperimetric problem. We aim to find the function $y(x)$, whose graph connects two given points $(a,A)$, $(b,B)$ on the $xy$-plane, with a prescribed arclength
        \begin{equation*}
            l = \int_a^b\sqrt{1+|y'(x)|^2}\dd{x}
        \end{equation*}
        such that the area between the graph and the $x$-axis is the largest. The functional in consideration is
        \begin{equation*}
            J[y] = \int_a^by(x)\dd{x}
        \end{equation*}
        with constraint
        \begin{equation*}
            R[y] = \int_a^b\sqrt{1+|y'(x)|^2}\dd{x} = l
        \end{equation*}
        Write down the Euler-Lagrange equation involving the multiplier $\lambda$ and show that the solution must be a part of a circle.
    \end{enumerate}
\end{enumerate}




\end{document}