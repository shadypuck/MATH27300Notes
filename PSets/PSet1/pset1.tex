\documentclass[../psets.tex]{subfiles}

\pagestyle{main}
\renewcommand{\leftmark}{Problem Set \thesection}
\setenumerate[1]{label={\textbf{\arabic*.}}}
\setenumerate[2]{label={(\arabic*)}}

\begin{document}




\section{IVP Examples and Physical Problems}
\subsection*{Required Problems}
\begin{enumerate}
    \item \marginnote{10/12:}Classify the following ordinary differential equations (systems) by indicating the order, if they are linear, and if they are autonomous.
    \begin{enumerate}
        \item $y'(x)+y(x)=0$.
        \begin{proof}[Answer]
            {\color{white}hi}
            \begin{center}
                \small
                \renewcommand{\arraystretch}{1.2}
                \begin{tabular}{c|c|c}
                    Order & Linear? & Autonomous?\\
                    \hline
                    1 & Yes & Yes\\
                \end{tabular}
            \end{center}
        \end{proof}
        \item $y''(t)=t\sin(y(t))$.
        \begin{proof}[Answer]
            {\color{white}hi}
            \begin{center}
                \small
                \renewcommand{\arraystretch}{1.2}
                \begin{tabular}{c|c|c}
                    Order & Linear? & Autonomous?\\
                    \hline
                    2 & No & No\\
                \end{tabular}
            \end{center}
        \end{proof}
        \item $x'=-y$, $y'=2x$.
        \begin{proof}[Answer]
            {\color{white}hi}
            \begin{center}
                \small
                \renewcommand{\arraystretch}{1.2}
                \begin{tabular}{c|c|c}
                    Order & Linear? & Autonomous?\\
                    \hline
                    1 & Yes & Yes\\
                \end{tabular}
            \end{center}
        \end{proof}
        \item $y'(t)=y(t)\sin(t)+\cos(y(t))$.
        \begin{proof}[Answer]
            {\color{white}hi}
            \begin{center}
                \small
                \renewcommand{\arraystretch}{1.2}
                \begin{tabular}{c|c|c}
                    Order & Linear? & Autonomous?\\
                    \hline
                    1 & Yes & No\\
                \end{tabular}
            \end{center}
        \end{proof}
    \end{enumerate}
    \item Transform the following differential equations to first-order systems.
    \begin{enumerate}
        \item $y^{(3)}+2y''-y'+y=0$.
        \begin{proof}
            Let
            \begin{equation*}
                x =
                \begin{pmatrix}
                    y\\
                    y'\\
                    y''\\
                \end{pmatrix}
            \end{equation*}
            Then
            \begin{equation*}
                x' =
                \begin{pmatrix}
                    y'\\
                    y''\\
                    y^{(3)}\\
                \end{pmatrix}
            \end{equation*}
            so, by comparing components between the above two vectors and then using the original linear equation to define the last entry (with substitutions), we obtain
            \begin{empheq}[box=\fbox]{align*}
                x_1' &= x_2\\
                x_2' &= x_3\\
                x_3' &= -2x_3+x_2-x_1
            \end{empheq}
        \end{proof}
        \item $x''-t\sin x'=x$.
        \begin{proof}
            In an analogous manner to the above, we can determine that
            \begin{empheq}[box=\fbox]{align*}
                y_1' &= y_2\\
                y_2' &= y_1+t\sin y_2
            \end{empheq}
        \end{proof}
    \end{enumerate}
    \item Solve the following differential equations with initial value $x(0)=x_0$. Also identify the set of $x_0$ for which these solutions are extendable to the whole of $t\geq 0$. When a solution cannot be extended to the whole of $t\geq 0$, determine its lifespan in terms of $x_0$.\par
    \emph{Example}: Solve $x'=x^2$ with $x(0)=x_0$. By separation of variables, the solution reads
    \begin{equation*}
        \int_{x_0}^x\frac{\dd{w}}{w^2} = \int_0^t\dd\tau
    \end{equation*}
    where the integral on the left-hand side cannot pass through $w=0$. The result is
    \begin{equation*}
        -\frac{1}{x}+\frac{1}{x_0} = t
        \quad\Longleftrightarrow\quad
        x(t) = \frac{x_0}{1-x_0t}
    \end{equation*}
    When $x_0\leq 0$, the solution exists throughout $t\geq 0$. When $x_0>0$, the solution only exists in $[0,1/x_0)$.
    \begin{enumerate}
        \item $x'=x\sin t$.
        \begin{proof}
            By separation of variables, the solution reads
            \begin{equation*}
                \int_{x_0}^x\frac{\dd{w}}{w} = \int_0^t\sin\tau\dd\tau
            \end{equation*}
            The result is
            \begin{equation*}
                \ln\frac{x}{x_0} = 1-\cos t
                \quad\Longleftrightarrow\quad
                \boxed{x(t) = x_0\e[1-\cos t]}
            \end{equation*}
            The set of $x_0$ for which this solution is extendable to the whole of $t\geq 0$ is $\boxed{\R}$.
        \end{proof}
        \item $x'=t^2\tan x$.
        \begin{proof}
            By separation of variables, the solution reads
            \begin{equation*}
                \int_{x_0}^x\cot w\dd{w} = \int_0^t\tau^2\dd\tau
            \end{equation*}
            where the integral on the left-hand side cannot pass through $x=\pi n$ for any $n\in\Z$. The result is
            \begin{equation*}
                \ln\left| \frac{\sin x}{\sin x_0} \right| = \frac{t^3}{3}
                \quad\Longleftrightarrow\quad
                \boxed{x(t) = \arcsin(\e[t^3/3]\sin x_0)}
            \end{equation*}
            The set of $x_0$ for which the solution is extendable to the whole of $t\geq 0$ is $\boxed{\emptyset}$ because $\cot(x)$ blows up periodically. When $x_0=\pi n$ for any $n\in\Z$, there is no solution because cotangent is undefined at these values and the improper integral blows up. When $x_0\neq\pi n$, the solution only exists in
            \begin{equation*}
                \boxed{\left[ 0,\sqrt[3]{3\ln\left| \frac{1}{\sin(x_0)} \right|} \right)}
            \end{equation*}
        \end{proof}
        \item $x'=1+x^2$.
        \begin{proof}
            By separation of variables, the solution reads
            \begin{equation*}
                \int_{x_0}^x\frac{1}{1+w^2}\dd{w} = \int_0^t\dd\tau
            \end{equation*}
            The result is
            \begin{equation*}
                \tan(x)-\tan(x_0) = t
                \quad\Longleftrightarrow\quad
                \boxed{x(t) = \arctan(t+\tan(x_0))}
            \end{equation*}
            The set of $x_0$ for which the solution is extendable to the whole of $t\geq 0$ is
            \begin{equation*}
                \boxed{\R\setminus\left\{ \frac{\pi}{2}+\pi n\mid n\in\Z \right\}}
            \end{equation*}
        \end{proof}
        \item $x'=\e[x]\sin t$.
        \begin{proof}
            By separation of variables, the solution reads
            \begin{equation*}
                \int_{x_0}^x\e[-w]\dd{w} = \int_0^t\sin\tau\dd\tau
            \end{equation*}
            The result is
            \begin{equation*}
                -\e[-x]+\e[-x_0] = 1-\cos t
                \quad\Longleftrightarrow\quad
                \boxed{x(t) = -\ln(\e[-x_0]-1+\cos t)}
            \end{equation*}
            The set of $x_0$ for which the solution is extendable to the whole of $t\geq 0$ is
            \begin{equation*}
                \boxed{\{x_0\in\R\mid x_0<\ln(1/2)\}}
            \end{equation*}
            When $x_0\geq\ln(1/2)$, the solution only exists in
            \begin{equation*}
                \boxed{\left[ 0,\arccos(1-\e[-x_0]) \right)}
            \end{equation*}
        \end{proof}
    \end{enumerate}
    \item Consider the harmonic oscillator equation, as mentioned in class:
    \begin{equation*}
        x''+\mu x'+\omega^2x = 0
    \end{equation*}
    Here, the initial data $x(0)=x_0$ and $x'(0)=x_1$ are real numbers.
    \begin{enumerate}
        \item Derive two linearly independent \emph{real} solutions when $\mu>0$. (Hint: You should consider the cases $\mu<2\omega$ and $\mu>2\omega$ separately.)
        \begin{proof}
            We first state and prove the following claim: If $r$ is a zero of the characteristic polynomial $r^2+ar+b=0$, then $\e[rx]$ is a solution to the ODE $y''+ay'+by=0$. The proof is simple --- plugging $y=\e[rx]$ and its derivatives $y'=r\e[rx]$ and $y''=r^2\e[rx]$ into the original ODE, we have that
            \begin{equation*}
                r^2\e[rx]+ar\e[rx]+b\e[rx] = (r^2+ar+b)\e[rx]
                = 0
            \end{equation*}
            iff $r^2+ar+b=0$, i.e., if $r$ is a root of said polynomial, as desired.\par
            With this guiding idea, we will find the roots of
            \begin{equation*}
                r^2+\mu r+\omega^2 = 0
            \end{equation*}
            Using the quadratic formula, the two roots are
            \begin{align*}
                r_1 &= \frac{-\mu+\sqrt{\mu^2-4\omega^2}}{2}&
                r_2 &= \frac{-\mu-\sqrt{\mu^2-4\omega^2}}{2}
            \end{align*}
            We now divide into two cases ($\mu>2\omega$ and $\mu<2\omega$). If $\mu>2\omega$, then $r_1,r_2$ are real and we take
            \begin{equation*}
                \boxed{\e[r_1t],\e[r_2t]}
            \end{equation*}
            to be our linearly independent, real solutions.\par
            On the other hand, if $\mu<2\omega$, then $r_1,r_2$ are of the form $\alpha\pm i\beta$. However, we can still obtain real solutions from these by taking the following linear combinations.
            \begin{align*}
                s_1 &= r_1+r_2 = 2\alpha&
                s_2 &= i(r_1-r_2) = 2\beta
            \end{align*}
            Thus, we take
            \begin{equation*}
                \boxed{\e[s_1t],\e[s_2t]}
            \end{equation*}
            to be our linearly independent, real solutions.\par
            Thus, our general solution is of the form
            \begin{equation*}
                x(t) = A\e[c_1t]+B\e[c_2t]
            \end{equation*}
            where $c_1=r_1,s_1$ and $c_2=r_2,s_2$ for some $A,B\in\R$. Plugging in the initial conditions, we get
            \begin{align*}
                x_0 = x(0) &= A+B\\
                x_1 = x'(0) &= Ac_1+Bc_2
            \end{align*}
            which we can solve for $A,B$, yielding
            \begin{equation*}
                \begin{cases}
                    A = \frac{x_1-x_0c_2}{c_1-c_2}\\
                    B = \frac{x_0c_1-x_1}{c_1-c_2}
                \end{cases}
            \end{equation*}
            Therefore, our final particular solution is
            \begin{equation*}
                \boxed{x(t) = \frac{x_1-x_0c_2}{c_1-c_2}\e[c_1t]+\frac{x_0c_1-x_1}{c_1-c_2}\e[c_2t]}
            \end{equation*}
        \end{proof}
        \item Recall that $\mu=b/m$ and $\omega^2=k/m$. Recall also that the mechanical energy for the oscillator reads
        \begin{equation*}
            E = \frac{1}{2}m|x'|^2+\frac{1}{2}kx^2
        \end{equation*}
        Compute the time derivative of $E$ and conclude that $E$ is exponentially decaying for $b>0$, i.e., the mechanical energy is not conserved in this case. Does this violate the law of conservation of mechanical energy?
        \begin{proof}
            % From part (a), we have that
            % \begin{align*}
            %     x(t) &= A\e[c_1t]+B\e[c_2t]&
            %     x'(t) &= Ac_1\e[c_1t]+Bc_2\e[c_2t]
            % \end{align*}
            % Thus,
            % \begin{equation*}
            %     E = \frac{1}{2}m\left| Ac_1\e[c_1t]+Bc_2\e[c_2t] \right|^2+\frac{1}{2}k\left( A\e[c_1t]+B\e[c_2t] \right)^2
            % \end{equation*}
            % Applying the chain rule, we have that
            % \begin{align*}
            %     \dv{E}{t} &= m\left( Ac_1\e[c_1t]+Bc_2\e[c_2t] \right)\left( Ac_1^2\e[c_1t]+Bc_2^2\e[c_2t] \right)+k\left( A\e[c_1t]+B\e[c_2t] \right)\left( Ac_1\e[c_1t]+Bc_2\e[c_2t] \right)\\
            %     &= mx'x''+kxx'
            % \end{align*}


            Applying the chain rule, we have that
            \begin{equation*}
                \dv{E}{t} = mx'x''+kxx'
            \end{equation*}
            It follows that
            \begin{align*}
                \dv{E}{t} &= mx'(-\mu x'-\omega^2x)+kxx'\\
                &= x'(-bx'-kx)+kxx'\\
                &= -b(x')^2
            \end{align*}
            Now $x'\neq 0$ (as an exponential function). Hence, $(x')^2>0$. This and $b>0$ show that $\dv{E}{t}$ is always equal to a negative value. But this is characteristic of exponential decay, as desired.\par
            Mechanical energy is conserved; it is dispersed from system to surroundings by the drag $b$.
        \end{proof}
    \end{enumerate}
    \item Use the transformation $y=tw$ to convert
    \begin{equation*}
        y' = f(y/t)
    \end{equation*}
    to an ODE in $w$. Write down this equation for $w$. Use this transformation to solve
    \begin{equation*}
        tyy'+4t^2+y^2 = 0
        ,\quad
        y(2) = -7
    \end{equation*}
    Determine the lifespan (you can use a calculator for an approximate value).
    \begin{proof}
        If $y=tw$, then
        \begin{equation*}
            \dv{y}{t} = w+t\dv{w}{t}
        \end{equation*}
        Thus, the ODE in terms of $w$ is
        \begin{equation*}
            \boxed{\dv{w}{t} = \frac{f(w)-w}{t}}
        \end{equation*}
        which is a separable differential equation.\par
        We have that
        \begin{equation*}
            tyy'+4t^2+y^2 = 0
            \quad\Longleftrightarrow\quad
            y' = -4\left( \frac{y}{t} \right)^{-1}-\frac{y}{t}
        \end{equation*}
        Using the above transformation yields
        \begin{equation*}
            \dv{w}{t} = \frac{(-4w^{-1}-w)-w}{t}
        \end{equation*}
        Transforming the initial condition as well gives
        \begin{equation*}
            w(2) = \frac{y(2)}{2} = -\frac{7}{2}
        \end{equation*}
        We can simplify and solve the above as follows.
        \begin{align*}
            \frac{\dd{w}}{-4w^{-1}-2w} &= \frac{\dd{t}}{t}\\
            -\frac{1}{4}\int_{-7/2}^w\frac{2v\dd{v}}{v^2+2} &= \int_2^t\frac{\dd\tau}{\tau}\\
            -\frac{1}{4}\left[ \ln(w^2+2)-\ln(14.25) \right] &= \ln(\frac{t}{2})\\
            w &= \pm\frac{1}{t^2}\sqrt{228-2t^4}\\
            \Aboxed{y(t) &= -\frac{1}{t}\sqrt{228-2t^4}}
        \end{align*}
        Note that we pick the negative in the final step to fit the initial condition.\par
        The lifespan of $y(t)$ can be determined by calculating when $228-2t^4=0$. This occurs such that the lifespan is approximately
        \begin{equation*}
            \boxed{[0,3.27]}
        \end{equation*}
    \end{proof}
    \item Use the transformation $w=y^{1-\alpha}$ to convert Bernoulli's equation
    \begin{equation*}
        y'+p(t)y = q(t)y^\alpha
        ,\quad
        \alpha\neq 0,1
    \end{equation*}
    to an ODE in $w$. Write down this equation for $w$. Use this transformation to solve
    \begin{equation*}
        6y'-2y = ty^4
        ,\quad
        y(0) = -2
    \end{equation*}
    Determine the lifespan (you can use a calculator for an approximate value).
    \begin{proof}
        If $w=y^{1-\alpha}$, then
        \begin{align*}
            y &= w^{1/(1-\alpha)}&
            \dv{y}{t} &= \frac{w^{\alpha/(1-\alpha)}}{1-\alpha}\dv{w}{t}
        \end{align*}
        Thus, the ODE in terms of $w$ is
        \begin{equation*}
            \boxed{\frac{w^{\alpha/(1-\alpha)}}{1-\alpha}\dv{w}{t}+p(t)w^{1/(1-\alpha)} = q(t)w^{\alpha/(1-\alpha)}}
        \end{equation*}
        which is an exact differential equation.\par
        We have that
        \begin{equation*}
            6y'-2y = ty^4
            \quad\Longleftrightarrow\quad
            y'+\left( -\frac{1}{3} \right)y = \left( \frac{t}{6} \right)y^4
        \end{equation*}
        Using the above transformation yields
        \begin{equation*}
            -\frac{w^{-4/3}}{3}\dv{w}{t}-\frac{w^{-1/3}}{3} = \frac{tw^{-4/3}}{6}
        \end{equation*}
        We can simplify and evaluate the above as follows.
        \begin{align*}
            \frac{1}{3}w^{-4/3}\dv{w}{t}+\frac{1}{3}w^{-1/3} &= -\frac{t}{6}w^{-4/3}\\
            \dv{w}{t}+w &= -\frac{t}{2}\\
            \e[t]\dv{w}{t}+\e[t]w &= -\frac{t}{2}\e[t]\\
            \dv{t}(\e[t]w) &= -\frac{t}{2}\e[t]\\
            \e[t]w &= -\frac{1}{2}\int t\e[t]\dd{t}\\
            &= -\frac{1}{2}\e[t](t-1)+C\\
            w &= -\frac{1}{2}(t-1)+C\e[-t]\\
            y^{-3} &= -\frac{1}{2}(t-1)+C\e[-t]\\
            y &= \left[ -\frac{1}{2}(t-1)+C\e[-t] \right]^{-1/3}
        \end{align*}
        We now apply the initial condition.
        \begin{align*}
            \left[ -\frac{1}{2}(0-1)+C\e[-0] \right]^{-1/3} &= y(0)\\
            \left[ \frac{1}{2}+C \right]^{-1/3} &= -2\\
            C &= -\frac{5}{8}
        \end{align*}
        Therefore, the solution to the ODE in question is
        \begin{equation*}
            \boxed{y(t) = \left[ -\frac{1}{2}(t-1)-\frac{5}{8}\e[-t] \right]^{-1/3}}
        \end{equation*}
        The equation \fbox{does not have finite lifespan}.
    \end{proof}
    \item Show that
    \begin{equation*}
        (4bxy+3x+5)y'+3x^2+8ax+2by^2+3y = 0
    \end{equation*}
    is an exact equation, no matter what value $a,b$ take. Find the implicit relation satisfied by the solution $y(x)$ and $x$.
    \begin{proof}
        To show that an equation of the form $gy'+f=0$ is exact, it will suffice to confirm that
        \begin{equation*}
            \pdv{g}{x} = \pdv{f}{y}
        \end{equation*}
        Since the equation in question is of this form, we may evaluate directly:
        \begin{align*}
            \pdv{g}{x} &= 4by+3&
            \pdv{f}{y} &= 4by+3
        \end{align*}
        By transitivity, we have the desired result.\par
        We now want to find $F$ such that $\pdv*{F}{x}=f$ and $\pdv*{F}{y}=g$. Starting with the former constraint, we can determine that
        \begin{align*}
            F(x,y) &= \int(3x^2+8ax+2by^2+3y)\dd{x}\\
            &= x^3+4ax^2+2bxy^2+3xy+h(y)
        \end{align*}
        where $h(y)$ is a functional "constant" of integration. We now differentiate with respect to $y$.
        \begin{equation*}
            \pdv{F}{y} = 4bxy+3x+\dv{h}{y}
        \end{equation*}
        Knowing that $\pdv*{F}{y}=g$, we can use the above equation to solve for $h$ as follows.
        \begin{align*}
            4bxy+3x+5 &= 4bxy+3x+\dv{h}{y}\\
            \dv{h}{y} &= 5\\
            h(y) &= 5y
        \end{align*}
        Therefore, we know that
        \begin{equation*}
            \boxed{F(x,y) = x^3+4ax^2+2bxy^2+3xy+5y}
        \end{equation*}
    \end{proof}
    \item Let $a,b$ be constants. For Euler's equation
    \begin{equation*}
        t^2y''+aty'+by = f(t)
    \end{equation*}
    consider the transformation $w(\tau)=y(\e[\tau])$. What is the differential equation satisfied by $w(\tau)$? Use this transformation to solve
    \begin{equation*}
        2t^2y''+3ty'-15y=0
        ,\quad
        y(1) = 0
        ,\quad
        y'(1) = 1
    \end{equation*}
    \begin{proof}
        The differential equation satisfied by $w(\tau)$ is
    \end{proof}
    \item Suppose there is a capacitor with capacitance $C$ being charged by a battery of fixed voltage $V_0$. Suppose there is a resistor $R$ connected to $C$. Then the charge $Q(t)$ of the capacitor satisfies the differential equation
    \begin{equation*}
        RQ'(t)+\frac{Q(t)}{C} = V_0
    \end{equation*}
    This is the equation for an RC charging circuit.\par
    Find the explicit solution of this equation with $Q(0)=0$. Explain why the product $RC$ is important in determining the charging time. For $R=\SI{e3}{\ohm}$, $V_0=\SI{1}{\volt}$, $C=\SI{1}{\micro\farad}$, how much time does it take for the capacitor to be charged to 98\%? (You may use a calculator.)
    \begin{proof}
        We can evaluate the ODE as follows.
        \begin{align*}
            \dv{Q}{t}+\frac{1}{RC}Q &= V_0\\
            \e[t/RC]\dv{Q}{t}+\frac{1}{RC}\e[t/RC]Q &= \e[t/RC]V_0\\
            \dv{t}(Q\e[t/RC]) &= \e[t/RC]V_0\\
            Q\e[t/RC] &= RCV_0\e[t/RC]+C_1\\
            Q(t) &= RCV_0+C_1\e[-t/RC]
        \end{align*}
        We now apply the initial condition.
        \begin{align*}
            0 &= Q(0)\\
            &= RCV_0+C_1\\
            C_1 &= -RCV_0
        \end{align*}
        Therefore, the solution to the ODE in question is
        \begin{equation*}
            \boxed{Q(t) = RCV_0\left( 1-\e[-t/RC] \right)}
        \end{equation*}
        The product $RC$ (technically referred to as the time constant) is important in determining charging time because it is directly proportional to the rate of exponential charging. Indeed, if $RC$ doubles, the capacitor will take twice as long to charge (and vice versa, for example, if $RC$ halves).\par
        The amount of time it takes for the capacitor to charge to 98\% under the given conditions ($R=\SI{e3}{\ohm}$ and $C=\SI{e-6}{\farad}$) may be determined as follows.
        \begin{align*}
            0.98 &= 1-\e[-t/RC]\\
            t &= -RC\ln(0.02)\\
            \Aboxed{t &= \SI{3.9e-3}{\second}}
        \end{align*}
    \end{proof}
    \item A parachutist is falling from a plane. Suppose the parachute is opened at height $H$, when the falling velocity is $v_0$. Suppose that the air resistance exerted on the parachute is proportional to the square of the velocity with ratio $\eta$. Let the gravitational constant be $g$, and suppose that the total mass of the parachutist and the parachute is $m$. Write down the differential equation satisfied by the shift $x$, together with the initial conditions. Solve this IVP. What is the velocity as $t\to +\infty$? Can you derive the final velocity based on physical considerations?
    \begin{proof}
        For the sake of simplicity, we will write a one-dimensional differential equation corresponding to vertical displacement. Let's begin.\par
        When the parachutist is falling freely, there is only one (idealized) force acting on them: gravity ($F_g$). As soon as the parachute is opened, another force is added to the mix: drag ($F_d$). By Newton's second law, the net force is equal to the parachutist/parachute's mass times their acceleration. Taking a convention of upwards displacement being positive, we can thus write that
        \begin{equation*}
            \sum F_z = F_d-F_g = ma
        \end{equation*}
        Since $a=x''$, $F_g=g$, and $F_d=\eta v^2=\eta(x')^2$, the differential equation satisfied by the shift $x$ is
        \begin{equation*}
            mx'' = \eta(x')^2-g
        \end{equation*}
        Let the time at which the parachute is opened be $t=0$. Then the initial conditions are
        \begin{align*}
            x(0) &= H&
            x'(0) &= v_0
        \end{align*}
        To solve this IVP, we substitute $v=x'$ and evaluate the resulting first-order differential equation to start:
        \begin{align*}
            mv' &= \eta v^2-g\\
            \frac{\dd{v}}{v^2-g/\eta} &= \frac{\eta}{m}\dd{t}\\
            \int_{v_0}^v\frac{\dd{w}}{w^2-g/\eta} &= \int_0^t\frac{\eta}{m}\dd{\tau}\\
            \coth^{-1}(v)-\coth^{-1}(v_0) &= \frac{\eta}{m}t\\
            v &= \coth(\frac{\eta}{m}t+\coth^{-1}(v_0))
            % -\frac{1}{2\sqrt{g/\eta}}\ln(\frac{(v+\sqrt{g/\eta})(v_0-\sqrt{g/\eta})}{(v-\sqrt{g/\eta})(v_0+\sqrt{g/\eta})}) &= \frac{\eta}{m}t\\
            % \ln(\frac{(v+\sqrt{g/\eta})(v_0-\sqrt{g/\eta})}{(v-\sqrt{g/\eta})(v_0+\sqrt{g/\eta})}) &= -\frac{2t\sqrt{g\eta}}{m}\\
            % \frac{(v+\sqrt{g/\eta})(v_0-\sqrt{g/\eta})}{(v-\sqrt{g/\eta})(v_0+\sqrt{g/\eta})} &= \e[-2t\sqrt{g\eta}/m]\\
            % \frac{(v+\sqrt{g/\eta})}{(v-\sqrt{g/\eta})} &= \frac{(v_0+\sqrt{g/\eta})}{(v_0-\sqrt{g/\eta})}\e[-2t\sqrt{g\eta}/m]\\
            % v &= -\sqrt{g/\eta}\frac{1+\frac{(v_0+\sqrt{g/\eta})}{(v_0-\sqrt{g/\eta})}\e[-2t\sqrt{g\eta}/m]}{1-\frac{(v_0+\sqrt{g/\eta})}{(v_0-\sqrt{g/\eta})}\e[-2t\sqrt{g\eta}/m]}\\
            % v &= -\sqrt{g/\eta}\frac{\frac{(v_0-\sqrt{g/\eta})}{(v_0+\sqrt{g/\eta})}+\e[-2t\sqrt{g\eta}/m]}{\frac{(v_0-\sqrt{g/\eta})}{(v_0+\sqrt{g/\eta})}-\e[-2t\sqrt{g\eta}/m]}
        \end{align*}
        Assuming the velocities are greater than one (a reasonable assumption; if not, change units), the hyperbolic cotangent is perfectly acceptable to use here. Returning the substitution $v=x'$, we can determine that
        \begin{align*}
            x' &= \coth(\frac{\eta}{m}t+\coth^{-1}(v_0))\\
            \int_H^x\dd{z} &= \int_0^t\coth(\frac{\eta}{m}\tau+\coth^{-1}(v_0))\dd\tau\\
            x-H &= \frac{m}{\eta}\ln(\sinh(\frac{\eta}{m}t+\coth^{-1}(v_0))\sqrt{v_0^2-1})\\
            \Aboxed{x &= H+\frac{m}{\eta}\ln(\sinh(\frac{\eta}{m}t+\coth^{-1}(v_0))\sqrt{v_0^2-1})}
        \end{align*}
        The final velocity approaches $\boxed{1}$.
    \end{proof}
\end{enumerate}


\subsection*{Bonus Problems}
\begin{enumerate}
    \item \textbf{The Catenoid.} Suppose there are two metal rings of radius $a$ placed parallel to each other in an $xyz$-coordinate space, with the $x$-axis passing through their centers. Suppose these two rings are contained in the planes $x=l$ and $x=-l$, respectively. An axial symmetric soap film is spanned by these two rings. Suppose its shape is obtained by rotating the graph of the function $y=y(x)$ with respect to the $x$-axis. In order to attain a stable configuration, the surface area is supposed to be minimal among all such surfaces of revolution.
    \begin{enumerate}
        \item Write down the surface area functional in terms of $y(x)$, its derivative, and the boundary conditions for this variational problem.
        \item Derive the Euler-Lagrange equation and find the solution. The shape is called a \textbf{catenoid}.
        \item If the two rings are very far away from each other, i.e., $l$ is very large, will the catenoid still be of minimal area among all competing surfaces that span these two rings? You do not have to give a mathematically rigorous answer; just imagine the physical situation. (Hint: What about two distinct disks spanned by these two rings?)
    \end{enumerate}
    \item \textbf{A Formulation of the Isoperimetric Problem.} Recall from multivariable calculus that in order to find a local extremum of the function $f(x_1,\dots,x_n)$ under the constraint $g(x_1,\dots,x_n)=0$, we can introduce a parameter $\lambda$ called the \textbf{Lagrange multiplier} and find the stationary point of the function
    \begin{equation*}
        f(x_1,\dots,x_n)-\lambda g(x_1,\dots,x_n)
    \end{equation*}
    \begin{enumerate}
        \item Write down the equations that must be satisfied by the stationary point $(x_1,\dots,x_n)$ of the function $f-\lambda g$ with the parameter $\lambda$ involved.
        \item Use the Lagrange multiplier method to find the maxima and minima of $f(x,y)=x+y$ under the constraint $x^2+y^2=1$.
        \item Now let us generalize this method to functionals. If we aim to find the extrema of a functional
        \begin{equation*}
            J[y] = \int_a^bF(x,y(x),y'(x))\dd{x}
        \end{equation*}
        under the constraint
        \begin{equation*}
            R[y] = \int_a^bG(x,y(x),y'(x))\dd{x} = 0
        \end{equation*}
        where $F(x,z,w)$ and $G(x,z,w)$ are known functions, we can try to find the extrema of the functional
        \begin{equation*}
            J[y]-\lambda R[y]
        \end{equation*}
        first. What is the Euler-Lagrange equation satisfied by this extrema (with $\lambda$ involved)?
        \item Now let us consider a version of the isoperimetric problem. We aim to find the function $y(x)$, whose graph connects two given points $(a,A)$, $(b,B)$ on the $xy$-plane, with a prescribed arclength
        \begin{equation*}
            l = \int_a^b\sqrt{1+|y'(x)|^2}\dd{x}
        \end{equation*}
        such that the area between the graph and the $x$-axis is the largest. The functional in consideration is
        \begin{equation*}
            J[y] = \int_a^by(x)\dd{x}
        \end{equation*}
        with constraint
        \begin{equation*}
            R[y] = \int_a^b\sqrt{1+|y'(x)|^2}\dd{x} = l
        \end{equation*}
        Write down the Euler-Lagrange equation involving the multiplier $\lambda$ and show that the solution must be a part of a circle.
    \end{enumerate}
\end{enumerate}




\end{document}