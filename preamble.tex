\usepackage[margin=1in]{geometry}
\usepackage{csquotes}
\usepackage{fancyhdr}
\usepackage{xr}
\usepackage{marginnote}
\usepackage{enumitem}
\usepackage{scrextend}
\usepackage[bottom]{footmisc}
\usepackage[style=apa]{biblatex}
\usepackage{siunitx}
\usepackage{graphicx,tikz,circuitikz}
\usepackage{subcaption}
\usepackage{amsmath,amssymb,amsthm}
\usepackage{bm,physics,empheq,nicematrix,mathtools}
\usepackage{mhchem}
\usepackage[hidelinks]{hyperref}

\MakeOuterQuote{"}

\fancypagestyle{main}{
    \fancyhf{}
    \fancyhead[L]{\leftmark}
    \fancyhead[R]{MATH 27300}
    \fancyfoot[R]{Labalme\ \thepage}
}
\fancypagestyle{plain}{
    \fancyhead{}
    \renewcommand{\headrulewidth}{0pt}
}

\reversemarginpar

\setitemize[3]{label={\scriptsize$\blacksquare$}}
\setitemize[4]{label={\tikz[scale=0.06,baseline={(0,-0.14)}]{
    \draw [line width=0.3pt] (0,1) -- (1.2,0) -- (0,-1) -- (3.5,0) -- cycle;
    \fill (1.2,0) -- (0,-1) -- (3.5,0);
}}}

\deffootnotemark{\textsuperscript{\textup{[}\thefootnotemark\textup{]}}}
\deffootnote[1.8em]{0em}{0em}{\textsuperscript{\thefootnote}}

\addbibresource{../main.bib}
\DefineBibliographyStrings{english}{bibliography={References}}

\usetikzlibrary{decorations.pathmorphing}
\colorlet{rex}{red!80!black!90!orange!80}
\colorlet{grx}{green!50!black!90!yellow!80}
\colorlet{ret}{red!80!black!90!orange!10}
\colorlet{orx}{orange!80!black!90!yellow!80}
\colorlet{ort}{orange!80!black!90!yellow!10}
\colorlet{ylx}{yellow!80!black!90!orange!80}
\colorlet{ylt}{yellow!80!black!90!orange!10}
\colorlet{blx}{blue!90!green!80}
\colorlet{puz}{blx!50!rex!50!magenta!10}

\ctikzset{bipoles/length=0.8cm}

\DeclareMathOperator{\spn}{span}
\DeclareMathOperator{\im}{im}
\DeclareMathOperator{\trc}{trace}
\DeclareMathOperator{\Ree}{Re}
\DeclareMathOperator{\Imm}{Im}
\DeclareMathOperator{\Trc}{Tr}
\DeclareMathOperator{\id}{id}
\DeclareMathOperator{\diag}{diag}
\DeclareMathOperator{\sign}{sign}

\newtheorem{theorem}{Theorem}

\NiceMatrixOptions{cell-space-limits=1pt}

\newcommand{\N}{\mathbb{N}}
\newcommand{\Z}{\mathbb{Z}}
\newcommand{\Q}{\mathbb{Q}}
\newcommand{\R}{\mathbb{R}}
\newcommand{\C}{\mathbb{C}}

\newcommand{\e}[1][]{\text{e}^{#1}}
\newcommand{\cnc}[2][]{[\ce{#2}]_\text{#1}}
\newcommand{\inp}[2]{\left\langle{#1},{#2}\right\rangle}

\usepackage{subfiles}